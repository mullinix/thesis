
% Note that if you want something in single space you can go back and
% forth between single space and normal space by the use of \ssp and
% \nsp.  If you want doublespacing you can use \dsp.  \nsp is normally
% 1.5 spacing unless you use the doublespace option (or savepaper
% option)
%
%(FORMAT) Usually you don't want to mess with the spacing for your
%(FORMAT) final version.  If you think/know that the thesis template
%(FORMAT) and/or thesis style file is incorrect/incomplete & PLEASE
%(FORMAT) contact the maintainer.  THANK YOU!!!

\chapter{EQUATIONS OF MOTION}
\label{chap:eom}
The equations of motion of the rotating cantilever beam system are developed through the principles of variational approximation as covered in Appendix \ref{app:variations}. The discussion begins by describing the energy equations of the system, then applies variational approximation to formulate the approximating form and applies the discretizing interpolation for a single element two-node system. Finally, the system is expanded to $N$-dimensional form, and the system matrices are formulated. 

\section{Energy Equations}

The energy equations consist of kinetic energy, denoted by $T$, and potential energy, denoted by $V$. These are precisely the quantities discussed in Appendix \ref{app:variations}, p.\pageref{app:variations}. In this case, the beam is discretized as a coupled mass-spring system. The coupling is determined by the equations of motion resultant from integration, simplification, and interpolation. The system that shall be considered is the system discussed by Ref.~\cite{chung2002dynamic}. In this system, it is assumed that the beam is long compared to the height and thickness, so that shear deformations may be neglected. 

The beam is affixed to a central hub which is rotated at a frequency $\Omega$. The direction which is in-plane with the rotations and along the length of the beam is referred herein as the axial direction, labeled $u$. The direction which is in-plane with rotations and orthogonal to the $u$ direction is referred to as the chord-wise direction, and is labeled $v$. The out-of-plane direction which is orthogonal to both $u,v$ is referred to as the flap-wise direction and is labeled $w$.

\begin{figure}[ht!]
\caption{Cantilever Beam with Rotational Oscillation \cite{chung2002dynamic}}
\centering
\includegraphics[width=0.65\textwidth]{images/chung_yoo_cantilever_beam.pdf}
\end{figure}

\subsection{Stretch Coordinate Frame}
\label{subsec:stretch}
Any displacement will cause elongation or shortening of the beam, which shall be named stretch and assigned the variable $s$. The stretch is affected by all displacement coordinates, $u,v,w$. Consequently, accounting for stretch from displacements in the Cartesian reference frame becomes unwieldy to maintain. In order to remedy this, it is necessary to create a coordinate relation between the displacements and the stretch. This section follows closely to the work done in Ref.~\cite{lima2012thesis}.

Consider the following diagram. In this diagram, the $u$ direction has been replaced by a dummy variable $\eta$ in order to prevent confusion during the derivation of the stretch term, $s$. 

\begin{figure}[ht!]
\caption{Stretch Coordinate Relation Diagram \cite{lima2012thesis}}
\centering
\includegraphics[width=0.65\textwidth]{images/stretch_coords.eps}
\end{figure}

From Pythagoras, we have the length of the hypotenuse of the infinitesimal displacements $\text d\eta, \text dv, \text dw$ given as
\begin{equation}
\text dS = \sqrt{(\text d\eta)^2+(\text dv)^2+(\text dw)^2},
\label{eq:pythag.basic}
\end{equation}
and integrating both sides of Eqn.~\ref{eq:pythag.basic} yields
\begin{equation}
S = \int_0^{x+u} \sqrt{(\text d\eta)^2+(\text dv)^2+(\text dw)^2},
\label{eq:arclength.a}
\end{equation}
which is known as the arclength integral in $\mathbb R^3$. Notice that the integration is done with respect to $\eta$, and varies over 0 to $x+u$ as in the diagram. \emph{Note:} the arclength $S$ is not equivalent to stretch $s$. This integral is complicated to compute in this form, so a change of variable is introduced. Let
\begin{equation}
\phi = \eta - u
\end{equation}
so that
\begin{equation}
\text d\phi = \text d\eta - \text du,
\end{equation}
and
\begin{equation}
\text d\eta = \text d\phi + \text du.
\label{eq:pythag.subs.vars}
\end{equation}
The integration limits are converted by considering that at the center of rotation, the displacement will be zero in all directions, so that stretch is zero. For the upper limit, consider that $\eta = u+x$ so that $\phi = x$. 

Substituting Eqn.~\ref{eq:pythag.subs.vars} into Eqn.~\ref{eq:pythag.basic} yields
\begin{equation}
\text dS = \sqrt{(\text d\phi + \text du)^2+(\text dv)^2+(\text dw)^2}.
\end{equation} 
By the chain rule, this can be rewritten as
\begin{equation}
\text dS = \left[\left(\text d\phi + \frac{\partial u}{\partial\phi}\text d\phi\right)^2+\left(\frac{\partial v}{\partial\phi}\text d\phi\right)^2+\left(\frac{\partial w}{\partial\phi}\text d\phi\right)^2\right]^{1/2},
\end{equation}
factoring out $\text d\phi$, replacing the integration limits, and substitution into Eqn.~\ref{eq:arclength.a} yields
\begin{equation}
S = \int_0^x \left[\left(1 + \frac{\partial u}{\partial\phi}\right)^2+\left(\frac{\partial v}{\partial\phi}\right)^2+\left(\frac{\partial w}{\partial\phi}\right)^2\right]^{1/2}\text{ d}\phi.
\label{eq:arclength.b}
\end{equation}
The first term in Eqn.~\ref{eq:arclength.b} is expanded as
\begin{equation}
\left(1 + \frac{\partial u}{\partial\phi}\right)^2 = 1+2\frac{\partial u}{\partial\phi}+\left(\frac{\partial u}{\partial\phi}\right)^2,
\label{eq:arclength.expansion}
\end{equation}
and considering the infinitesimals,
\begin{equation}
1+2\frac{\partial u}{\partial\phi} \gg \left(\frac{\partial u}{\partial\phi}\right)^2,
\end{equation}
we approximate Eqn.~\ref{eq:arclength.expansion} as
\begin{equation}
\left(1 + \frac{\partial u}{\partial\phi}\right)^2 \approx 1+2\frac{\partial u}{\partial\phi}.
\label{eq:arclength.u.approx}
\end{equation}
Let
\begin{equation}
f(y) = (1+y)^{1/2}.
\end{equation}
The McLaurin expansion of $f$ is given as
\begin{eqnarray}
f(y) &=& f(0)+f'(0)y+\frac{1}{2}f''(0)y^2+\cdots{\vrule width 0in depth .1in}\nonumber \\
&=& (1+0)^{1/2}+\frac{1}{2}(1+0)^{-1/2}y-\frac{1}{4}(1+0)^{-3/2}y^2+\cdots{\vrule width 0in depth .1in}\nonumber \\
&=& 1+\frac{y}{2}-\frac{y^2}{4}+\cdots,
\end{eqnarray}
and assuming small variations so that $y^2\ll y$, we approximate $f$ near 0 by
\begin{equation}
f(y) \approx 1+\frac{y}{2}.
\label{eq:f_func.approx}
\end{equation}
Plugging Eqn.~\ref{eq:arclength.u.approx} in to Eqn.~\ref{eq:arclength.b} yields
\begin{equation}
S = \int_0^x \left[1+2\frac{\partial u}{\partial\phi}+\left(\frac{\partial v}{\partial\phi}\right)^2+\left(\frac{\partial w}{\partial\phi}\right)^2\right]^{1/2} \text{ d}\phi.
\end{equation}
Observe that this can be rewritten as 
\begin{equation}
S = \int_0^x \left[1+\left\lbrace 2\frac{\partial u}{\partial\phi}+\left(\frac{\partial v}{\partial\phi}\right)^2+\left(\frac{\partial w}{\partial\phi}\right)^2\right\rbrace\right]^{1/2} \text d\phi,
\label{eq:arclength.grouped}
\end{equation}
and for 
\begin{equation}
y = 2\frac{\partial u}{\partial\phi}+\left(\frac{\partial v}{\partial\phi}\right)^2+\left(\frac{\partial w}{\partial\phi}\right)^2,
\end{equation}
the approximation in Eqn.~\ref{eq:f_func.approx} may be applied to Eqn.~\ref{eq:arclength.grouped} so that arclength may be written as
\begin{equation}
S = \int_0^x 1+\frac{1}{2}\left[ 2\frac{\partial u}{\partial\phi}+\left(\frac{\partial v}{\partial\phi}\right)^2+\left(\frac{\partial w}{\partial\phi}\right)^2 \right]\text{ d}\phi.
\end{equation}
Distribution of the constant $1/2$ and integration over the first two terms yields
\begin{equation}
S = x + u + \frac{1}{2}\int_0^x\left(\frac{\partial v}{\partial\phi}\right)^2\text{ d}\phi+\frac{1}{2}\int_0^x\left(\frac{\partial w}{\partial\phi}\right)^2\text{ d}\phi.
\end{equation}
We now define the stretch term $s$ in relation to the arclength $S$ by
\begin{equation}
S = x+s,
\end{equation}
so that stretch can be written as
\begin{equation}
s = u + \frac{1}{2}\int_0^x\left(\frac{\partial v}{\partial\phi}\right)^2\text{ d}\phi+\frac{1}{2}\int_0^x\left(\frac{\partial w}{\partial\phi}\right)^2\text{ d}\phi.
\end{equation}
For ease of notation, let
\begin{eqnarray}
h_v &=& \frac{1}{2}\int_0^x\left(\frac{\partial v}{\partial\phi}\right)^2\text{ d}\phi{\vrule width 0in depth .1in}\\
h_w &=& \frac{1}{2}\int_0^x\left(\frac{\partial w}{\partial\phi}\right)^2\text{ d}\phi
\end{eqnarray}
so that 
\begin{equation}
s = u+h_v+h_w.
\label{eq:stretch.full}
\end{equation}
The time rate of change of stretch can thus be expressed as 
\begin{equation}
\dot s = \dot u + \dot h_v+\dot h_w,
\label{eq:stretch.full.dt}
\end{equation}
where the chain rule yields
\begin{eqnarray}
\dot h_v &=& \int_0^x\frac{\partial v}{\partial\phi}\frac{\partial \dot v}{\partial\eta}\text{ d}\phi{\vrule width 0in depth .1in}\\
\dot h_w &=& \int_0^x\frac{\partial w}{\partial\phi}\frac{\partial \dot w}{\partial\eta}\text{ d}\phi.
\end{eqnarray}

With the book-keeping surrounding stretch completed, it is now possible to ascertain the equations of motion. Note, however, that there were many assumptions that will require the numerics to include a sufficiently fine grid in both time and space such that the assumptions are valid. A discussion of what ``sufficiently fine'' pertains to is given in Section~\ref{subsec:discretizing}.

\subsection{Kinetic Energy}
The system's kinetic energy is given by the classical scalar kinetic energy formulation
\begin{equation}
T = \frac{1}{2}m\vec{\textbf{v}}^2.
\label{eq:kin.scalar}
\end{equation}

The generalized integral representation of this equation is given by

\begin{equation}
T = \frac{1}{2}\int_{0}^{L}\rho A \vec{\textbf{v}}\cdot\vec{\textbf{v}}\text{ d}x,
\label{eq:kin.int.full}
\end{equation}
where $L$ is the length of the cantilever beam, $\rho$ and $A$ are the point density and cross section area, respectively. The units for $\rho$ are $\frac{kg}{m^3}$, and $A$ is $m^2$, and since Equation \ref{eq:kin.int.full} integrates upon a singular axis, we recover mass along a single axis by the product $\rho A$ at each point along the beam.

If density and cross section are uniform along the entire beam, Equation \ref{eq:kin.int.full} can be written as

\begin{equation}
T = \frac{1}{2}\rho A\int_{0}^{L}\vec{\textbf{v}}\cdot\vec{\textbf{v}}\text{ d}x,
\label{eq:kin.int.iso}
\end{equation}
where for each point on the beam, Equation \ref{eq:kin.int.iso} is precisely the same as Equation \ref{eq:kin.scalar}.

From the rotation, and deformation corresponding to bending, the velocity at the point $P$ is

\begin{equation}
\vec{\textbf{v}}_p = (\dot u-\Omega v)\textbf{i}+[\dot v+\Omega(a+x+u)]\textbf{j}+ \dot w\textbf{k}.
\label{eq:vel.vec}
\end{equation}
Rotation induces angular momentum, and allowing for stretch along the axial direction yields that the momentum from the rotational speed $\Omega$ is calculated by

\begin{equation}
L = \Omega\cdot(a+x+u) A\rho,
\label{eq:ang.mom}
\end{equation}
where $(a+x)$ is the point position along the axial direction when the system is at rest, $u$ is the displacement along the axial direction when the system is rotating at frequency $\Omega$.

The corresponding direction of rotation is the same direction as $v$, so that the chord-wise component of velocity is increased by this momentum. Assuming conserved momentum, a change in the chord-wise direction must result in a change in the axial direction. Then an increase in momentum in the chord-wise direction must result in a decrease in momentum in the axial direction, so that the $-\Omega v$ term in Equation \ref{eq:vel.vec} is required.  The flap-wise direction is unaffected, and it is not present in Equation \ref{eq:ang.mom}. The assumption of conserved momentum is valid since the system achieves stability; otherwise, the momentum would grow rapidly and the beam would break. 

Plugging Equation \ref{eq:vel.vec} into Equation \ref{eq:kin.int.iso} yields
\begin{align}
T &= \frac{1}{2}\rho A\int_{0}^{L}\left\langle(\dot u-\Omega v)\textbf{i}+[\dot v+\Omega(a+x+u)]\textbf{j}+ \dot w\textbf{k}\right\rangle^2\text{ d}x, \\
  &= \frac{1}{2}\rho A\int_{0}^{L} (\dot u-\Omega v)^2+[\dot v+\Omega(a+x+u)]^2+\dot w^2\text{ d}x.
  \label{eq:kin.final}
\end{align}

\begin{comment}
\textbf{BEGIN BROKEN DOWN POTENTIAL}

Substituting $u,\dot u$ from Eqns.~\ref{eq:stretch.full} and \ref{eq:stretch.full.dt} yields
\begin{equation}
T = \frac{1}{2}\rho A\int_{0}^{L} (\dot s-(\dot h_v+\dot h_w)-\Omega v)^2+[\dot v+\Omega(a+x+s-(h_v+h_w))]^2+\dot w^2\text{ d}x.
\end{equation}

The expanded form of Eqn.~\ref{eq:stretch.full.dt} can be broken down term-wise. For ease of notation, let
\begin{align}
h &= h_v+h_w,\label{eq:h.subs}\\
\dot h &= \dot h_v+\dot h_w.\label{eq:hdot.subs}
\end{align}
The first term expands as
\begin{align}
(\dot s-\dot h-\Omega v)^2 &= \dot s^2+\dot h^2+\Omega^2 v^2-2\dot s\dot h\\
&-2\dot s\Omega v+2\Omega \dot h v,\nonumber
\end{align}
and the second term expands as
\begin{align}
[\dot v+\Omega(a+x+s-h)]^2 &= \dot v^2 +\Omega^2(a+x+s-h)^2 \\
& +2\dot v\Omega(a+x+s-h).\nonumber
\end{align}
Further expansion yields
\begin{align}
[\dot v+\Omega(a+x+s-h)]^2 &= \dot v^2 +\Omega^2((a+x+s)^2+h^2) \nonumber\\
& -2\Omega^2(a+x+s)h\\
& +2\dot v\Omega(a+x+s-h).\nonumber
\end{align}
Note that there are higher order powers of $h,\dot h$; since $h, \dot h$ are small by assumption, we neglect these terms, and write the expansions as
\begin{align}
(\dot s-\dot h-\Omega v)^2 &= \dot s^2+\Omega^2 v^2-2\dot s\dot h-2\dot s\Omega v+2\Omega \dot h v,\\
[\dot v+\Omega(a+x+s-h)]^2 &= \dot v^2 +\Omega^2(a+x+s)^2 -2\Omega^2(a+x+s)h\\
& +2\dot v\Omega(a+x+s-h)\nonumber
\end{align}
Keeping the compact notation of $h,\dot h$ and substituting the expansions into Eqn.~\ref{eq:stretch.full.dt} yields the kinetic energy equation
\begin{align}
T &= \frac{1}{2}\rho A\int_{0}^{L} \dot s^2-2\dot s(\dot h+\Omega v)+\dot v^2+2\dot v\Omega(a+x+s-h)\\
&\qquad+2\Omega \dot h v +\Omega^2( v^2+(a+x+s)^2 -2(a+x+s)h)+\dot w^2\text{d}x.\nonumber
\end{align}
\textbf{END BROKEN DOWN POTENTIAL}
\end{comment}

\subsection{Potential Energy}

The potential energy of the system is taken to be strain energy. This is because when deformed, the beam will try to restore itself to its original shape; the restoring force can be thought of as stored energy, and typically this is called strain energy. This assumption is taken because of the properties of the material that shall be considered, typically metals such as steel and aluminum.

\subsubsection{Building the Strain Equations}
By considering an applied force to the beam, and applying Equation \ref{eq:stress.force} to Equation \ref{eq:strain.energy.final}, we rewrite the total strain energy as
\begin{equation}
U = \int_0^L\int_A\frac{1}{2E}\left(\frac{F}{A}\right)^2\text{ d}A\text{ d}x,
\end{equation}
and integrating over the area $A$ yields
\begin{equation}
U = \int_0^L\frac{F^2}{2EA}\text{ d}x.
\end{equation}
The axial stress resultant from stretch can be expressed by considering the stress-strain relationship in Equation \ref{eq:strain.hooke}, and applying the strain definition from Equation \ref{eq:strain.oned}, we write
\begin{equation}
\sigma = E\frac{\partial s}{\partial x}
\end{equation} 
thus
\begin{equation}
U_{\text{stretch}} = \int_0^L\frac{EA}{2}\left(\frac{\partial s}{\partial x}\right)^2\text{ d}x.
\end{equation}

The strain induced by bending is given by
\begin{equation}
\varepsilon = -\frac{y}{\rho},
\label{eq:strain.bend.basic}
\end{equation}
where $y$ is the distance from the neutral plane. For axisymmetric materials, maximum strain occurs at the boundary (or for a cylinder, the radius). $\rho$ is given as the radius of curvature; that is, the radius of the osculating circle that best fits the curvature of the bend. The radius of curvature in 2D orthogonal (or cartesian) coordinates is expressed as the arclength of the path traveling along the rod.  The arclength integral is computed to be 
\begin{equation}
\rho = \left|\frac{(1+{y'}^2)^{3/2}}{y''}\right|.
\end{equation}
Here, $y'$ is the derivative of $y$ with respect to $x$. We now assume the bending is small so that $y'\ll1$ and approximate the radius of curvature to be
\begin{equation}
\rho = \frac{1}{\frac{\text d^2y}{\text dx^2}}.
\label{eq:radius.curvature}
\end{equation}
This assumption is valid for the metallic alloys we are considering as a result of material properties. Additionally, we are in a coordinate frame in which we will consider the directions $u,v,w$ with respect to $x$.

Substituting Eqn.~\ref{eq:radius.curvature} into Eqn.~\ref{eq:strain.bend.basic}, and adjusting into the flap-wise coordinate frame yields
\begin{equation}
\varepsilon = -y\left(\frac{\partial^2v}{\partial x^2}\right),
\end{equation}
and using the stress-strain relationship given in Eqn.~\ref{eq:strain.hooke}, we find
\begin{equation}
\sigma = -Ey\left(\frac{\partial^2v}{\partial x^2}\right).
\label{eq:stress.bending.early}
\end{equation}
Bending moment, $M$, is defined to be 
\begin{equation}
M = -\int_Ay\sigma \text{ d}A = \int_AEy^2\left(\frac{\partial^2v}{\partial x^2}\right)\text{ d}A
\end{equation}
For each subsection along $x$, both $E$ and $\frac{\partial^2v}{\partial x^2}$ are considered constant with respect to the area, and we write the integral as
\begin{equation}
M = E\left(\frac{\partial^2v}{\partial x^2}\right)\int_Ay^2\text{ d}A.
\label{eq:moment.inertia}
\end{equation}
The quantity resultant of the area integral is known as the \emph{moment of inertia}, and in this case it is the second moment about $z$, written as
\begin{equation}
I_{zz} = \int_Ay^2\text{ d}A.
\end{equation}
With this simplification, we write the bending moment as
\begin{equation}
M_z = EI_{zz}\left(\frac{\partial^2v}{\partial x^2}\right).
\end{equation}
Upon substitution for $\frac{\partial^2v}{\partial x^2}$, we may rewrite Eqn.~\ref{eq:stress.bending.early} as
\begin{equation}
\sigma = \frac{M_zy}{I_{zz}}.
\label{eq:stress.bending.full}
\end{equation}
Plugging this stress into the strain integral from Eqn.~\ref{eq:strain.energy.final} we have
\begin{equation}
U_{\text{bending}} = \int_0^L\int_A\frac{1}{2}\frac{\sigma^2}{E}\text{ d}A\text{ d}x = \int_0^L\int_A\frac{1}{2E}\left(\frac{M_zy}{I_{zz}}\right)^2\text{ d}A\text{ d}x
\end{equation}
Distributing the power, we use the principle from Eqn.~\ref{eq:moment.inertia} to pull out the terms not involving $y$, and integrating over the cross-section area cancels one moment of inertia term $I_{zz}$, which yields
\begin{equation}
U_{\text{bending}} = \int_0^L\frac{M_z^2}{2EI_{zz}}\text{ d}x,
\end{equation}
and substituting for $M_z$ yields
\begin{equation}
U_{\text{bending}} = \int_0^L\frac{EI_{zz}}{2}\left(\frac{\partial^2 v}{\partial x^2}\right)^2\text{ d}x.
\label{eq:strain.bending.final}
\end{equation}

This term holds for bending in two directions, flap-wise ($v$) and chord-wise ($w$). In the case of chord-wise bending, the bending moment is written as $M_y$, the moment of inertia is $I_{yy}$, and the inverse radius of curvature is given as $\frac{\partial^2w}{\partial x^2}$. Otherwise, the formulation is precisely the same.

The full strain energy equation is a combination of these terms, given as
\begin{equation}
U = \int_0^L\frac{E}{2}\left[I_{zz}\left(\frac{\partial^2 v}{\partial x^2}\right)^2+I_{yy}\left(\frac{\partial^2 w}{\partial x^2}\right)^2+A\left(\frac{\partial s}{\partial x}\right)^2\right]\text{ d}x.
\label{eq:strain.final.chooyung}
\end{equation}

\section{Applying Variational Approximation}
Now that the kinetic and potential energy equations have been determined, the principles of variational approximation may be used to approximate the equations of motion. We begin by writing down the Lagrangian of the system, then rearranging the system by way of integration by parts to develop the Euler-Lagrange equations for the system. The non-vanishing terms that are not included in the Euler-Lagrange formulation are used to formulate approximation functions which both satisfy the boundary conditions and vanish as required to permit the use of variational approximation. Additionally, the properties of the physical system are considered when choosing the class of test functions which might be appropriate.

\subsection{Developing the Euler-Lagrange Equations}
The Lagrangian is given to be 
\begin{equation}
L = T-V = T-U.
\end{equation}
Applying variations as in Appendix~\ref{app:variations}, we have the functional 
\begin{equation}
J = \int_{t_0}^{t_1}L\text{ d}t
\end{equation}
to be minimized as
\begin{equation}
\delta J = \delta\int_{t_0}^{t_1}L\text{ d}t.
\end{equation}
By the linearity of $\delta$ we may move the variation inside the integral as
\begin{equation}
\delta J = \int_{t_0}^{t_1}\delta L\text{ d}t = \int_{t_0}^{t_1}\delta (T-U)\text{ d}t = \int_{t_0}^{t_1}\delta T - \delta U\text{ d}t.
\end{equation}

For $T,U$, each has a spatial integral. We define the subfunctions $\hat T, \hat U$ as
\begin{eqnarray}
T = \int_0^L \hat T \text{ d}x, \\
U = \int_0^L \hat U \text{ d}x,
\end{eqnarray}
and write
\begin{equation}
\delta J = \int_{t_0}^{t_1}\int_{0}^{L}\delta\hat T - \delta\hat U\text{ d}x\text{ d}t.
\end{equation}
The chain rule for the variational operator requires that each independent variable of the function under variation is considered. For the kinetic energy, we have that
\begin{equation}
\hat T = \hat T(u,\dot u,v,\dot v,w,\dot w),
\end{equation}
which yields 
\begin{equation}
\delta \hat T = \sum_{p_i}\frac{\partial \hat T}{\partial p_i}\delta p_i+\frac{\partial \hat T}{\partial \dot{p_i}}\delta \dot{p_i},\qquad p_i\in\{u,v,w\}.
\end{equation}
Note that there isn't explicitly a function $w$ in $\hat T$, but including this term for ease of notation is harmless since $\frac{\partial\hat T}{\partial w}=0$. For the potential energy function, we have that
\begin{equation}
\hat U = \hat U(s_x,v_{xx},w_{xx}),
\end{equation}
which yields 
\begin{equation}
\delta \hat U = \sum_{q_j}\frac{\partial \hat U}{\partial q_j}\delta q_j,\qquad q_j\in\{s_x,v_{xx},w_{xx}\}.
\end{equation}
In order to properly formulate the Lagrangian of the system, we must transform these coordinates so that terms are given in $\delta s,\delta v,\delta w$. Integration by parts shall be employed in various ways in order to derive the Lagrangian in this form.

We first introduce a tool to transform products by way of integration by parts. The assumptions are strict and specific, but are well suited for problems in this domain, since they always apply to cantilever beams.

\begin{lemma}
Let $f$ be an integrable function in the domain $[0,L]$ and define $F = \int f(x)\text{ d}x$ on $[0,L]$. If $F(0) = f(L) = 0$, then given the integrable function $g$,
\begin{equation}
I = \int_0^L\left(\int_0^xg(\xi)\text{ d}\xi\right)f(x)\text{ d}x = \int_0^L\left(\int_L^xf(\xi)\text{ d}\xi\right)g(x)\text{ d}x.
\end{equation}
\label{lem:switch.integrals}
\end{lemma}
\begin{proof}
We perform integration by parts as follows. Let
\[\alpha = \int_0^xg(\xi)\text{ d}\xi.\]
Then
\[\text{d}\alpha = g(x)\text{ d}x\]
by the Fundamental Theorem of Calculus. Let
\[\text{d}\beta = f(x)\text{ d}x.\]
Then
\[\beta = F(x).\]
Applying the integration by parts formula, we have
\begin{equation*}
\begin{array}{rcll}
I &=&  \left.\int_0^xg(\xi)\text{ d}\xi F(x)\right|_0^L - \int_0^LF(x)g(x)\text{ d}x & \\
  &=&  \int_0^Lg(x)\text{ d}xF(L) - \int_0^LF(x)g(x)\text{ d}x &\text{(since $F(0)=0$)} \\
  &=&  \int_0^LF(L)g(x)\text{ d}x - \int_0^LF(x)g(x)\text{ d}x &\text{(since $F(L)$ is constant)} \\
  &=&  \int_0^L[F(L)-F(x)]g(x)\text{ d}x &\text{(by linearity)} \\
  &=&  \int_0^L\left(\int_L^xf(\xi)\text{ d}\xi\right)g(x)\text{ d}x &\text{(by definition of $F$)} \\
\end{array}
\end{equation*}
\end{proof}

Since time, $t$ and space, $x$ are independent of each other, the iterated integrals may be switched, and we may write the variation on $\hat T$ as follows.
\begin{equation}
\int_{t_0}^{t_1}\int_0^L\delta\hat T\text{ d}x\text{ d}t = \int_0^L\int_{t_0}^{t_1}\delta\hat T\text{ d}t\text{ d}x.
\end{equation}
The term for $q_i=u$ has special properties and we shall consider those properties later; we first analyze $q_i\in\{v,w\}$. The $\dot q_i$ term is integrated by using integration by parts. This is accomplished by utilizing the time integral; the spatial integral shall be excluded for brevity.
First, write out the full variation on the kinetic energy:
\begin{equation}
\int_{t_0}^{t_1}\delta\hat T(\dot q_i,q_i)\text{ d}t = \int_{t_0}^{t_1}\frac{\partial\hat T}{\partial \dot q_i}\delta \dot q_i\text{ d}t+\int_{t_0}^{t_1}\frac{\partial\hat T}{\partial q_i}\delta q_i\text{ d}t.
\label{eq:kin.variation}
\end{equation}
Then, consider the first integral on the right hand side, and let
\begin{equation}
\alpha = \frac{\partial\hat T}{\partial \dot q_i},
\end{equation}
then
\begin{equation}
\text{ d}\alpha = \frac{\text d}{\text dt}\left(\frac{\partial\hat T}{\partial \dot q_i}\right)\text{ d}t.
\end{equation}
Next, let
\begin{equation}
\text{d}\beta = \delta \dot q_i\text{ d}t,
\end{equation}
then
\begin{equation}
\beta = \int \frac{\text d}{\text dt}(\delta q_i)\text{ d}t = \delta q_i.
\end{equation}
Finally, we may write the first integral as
\begin{equation}
\int_{t_0}^{t_1}\frac{\partial\hat T}{\partial \dot q_i}\delta \dot q_i\text{ d}t = \left.\frac{\partial\hat T}{\partial \dot q_i}\delta q_i\right |_{t_0}^{t_1}-\int_{t_0}^{t_1}\frac{\text d}{\text dt}\left(\frac{\partial\hat T}{\partial \dot q_i}\right)\delta q_i\text{ d}t.
\end{equation}
Variational approximation requires that
\begin{equation}
\left.\frac{\partial\hat T}{\partial \dot q_i}\delta q_i\right |_{t_0}^{t_1} = 0,
\end{equation}
thus the first integral is given by
\begin{equation}
\int_{t_0}^{t_1}\frac{\partial\hat T}{\partial \dot q_i}\delta \dot q_i\text{ d}t = -\int_{t_0}^{t_1}\frac{\text d}{\text dt}\left(\frac{\partial\hat T}{\partial \dot q_i}\right)\delta q_i\text{ d}t.
\label{eq:kin.variation.byparts}
\end{equation}
Substitution of Eqn.~\ref{eq:kin.variation.byparts} into Eqn.~\ref{eq:kin.variation} yields
\begin{equation}
\int_{t_0}^{t_1}\delta\hat T(\dot q_i,q_i)\text{ d}t = \int_{t_0}^{t_1}\left\lbrace-\frac{\text d}{\text dt}\left[\frac{\partial\hat T}{\partial \dot q_i}\right]+\frac{\partial\hat T}{\partial q_i}\right\rbrace\delta q_i\text{ d}t.
\end{equation}
We now reintroduce the spatial integral and switch the order of integration so that $\hat T,\hat U$ are integrating in the same order, which yields the following expression.
\begin{equation}
\int_{t_0}^{t_1}\int_0^L\delta\hat T(\dot q_i,q_i)\text{ d}x\text{ d}t = \int_{t_0}^{t_1}\int_0^L\left\lbrace-\frac{\text d}{\text dt}\left[\frac{\partial\hat T}{\partial \dot q_i}\right]+\frac{\partial\hat T}{\partial q_i}\right\rbrace\delta q_i\text{ d}x\text{ d}t.
\end{equation}

We now continue the process for strain energy.

The strain energy $U$ yields differing variation fields which must be converted to be in the $q_i$ form.

The dynamic expressions in $U$ are given in Eqn.~\ref{eq:strain.final.chooyung} as $v_{xx}$, $w_{xx}$, and $s_x$. The generalized coordinates are $q_{xx_i}$, and $q_{x_i}$, where the subscript $x$ represents the derivative with respect to $x$, and the subscript $i$ represents the $i$-th direction.

The chain rule applied to the strain term $U$ is given by
\begin{equation}
\delta U(q_{xx_i},q_{x_i}) = \frac{\partial U}{\partial q_{xx_i}}\delta q_{xx_i}+\frac{\partial U}{\partial q_{x_i}}\delta q_{x_i}.
\end{equation}
Once again, we aim to modify this expression to that the terms on the right are given as variations of $q_i$.
The full variation of the strain energy is given as
\begin{equation}
\int_{t_0}^{t_1}\delta U(q_{xx_i},q_{x_i})\text{ d}t = \int_{t_0}^{t_1}\frac{\partial U}{\partial q_{xx_i}}\delta q_{xx_i}\text{ d}t+\int_{t_0}^{t_1}\frac{\partial U}{\partial q_{x_i}}\delta q_{x_i}\text{ d}t.
\end{equation}
We now rewrite $U$ as follows.
\begin{equation}
U = \int_0^L \hat{U}\text{ d}x,
\end{equation}
so that
\begin{equation}
\delta U = \delta \int_0^L\hat{U}\text{ d}x = \int_0^L\delta\hat{U}\text{ d}x.
\end{equation}
The dynamic components of $U$ are the same as that of $\hat{U}$, so the chain rule applies in the same way and the full variation applied to this new expression is given as
\begin{equation}
\int_{t_0}^{t_1}\int_0^L\delta \hat U(q_{xx_i},q_{x_i})\text{ d}x\text{ d}t = \int_{t_0}^{t_1}\int_0^L\frac{\partial \hat U}{\partial q_{xx_i}}\delta q_{xx_i}\text{ d}x\text{ d}t+\int_{t_0}^{t_1}\int_0^L\frac{\partial \hat U}{\partial q_{x_i}}\delta q_{x_i}\text{ d}x\text{ d}t.
\label{eq:strain.byparts.initial}
\end{equation}
Following the same form, we integrate by parts. First, consider the first integral on the right side. For brevity, we shall not write the time integral, and focus on the spatial integral. As before, let
\begin{equation}
\alpha = \frac{\partial \hat U}{\partial q_{xx_i}}.
\end{equation}
Then
\begin{equation}
\text d\alpha = \frac{\text d}{\text dx}\left(\frac{\partial \hat{U}}{\partial q_{xx_i}}\right)\text{ d}x.
\end{equation}
Let
\begin{equation}
\text d\beta = \delta q_{xx_i}\text{ d}x.
\end{equation}
Then 
\begin{equation}
\beta = \int \frac{\text d}{\text dx}\delta q_{x_i}\text{ d}x = \delta q_{x_i}.
\end{equation}
Applying the integration by parts formula, we have
\begin{equation}
\int_0^L\frac{\partial \hat U}{\partial q_{xx_i}}\delta q_{xx_i}\text{ d}x = \left.\frac{\partial \hat U}{\partial q_{xx_i}}\delta q_{x_i}\right|_0^L-\int_0^L\frac{\text d}{\text dx}\left(\frac{\partial \hat{U}}{\partial q_{xx_i}}\right)\delta q_{x_i}\text{ d}x.
\label{eq:strain.byparts.term1.first}
\end{equation}
In this case, the first expression will not necessarily vanish. This provides the means for ensuring the approximating functions will satisfy the physical boundary conditions. Furthermore, the second term is still not in proper form, as the variation is acting on $q_{x_i}$, not $q_i$. Thus, we must integrate the second term by parts.
Let
\begin{equation}
\alpha = \frac{\text d}{\text dx}\left(\frac{\partial \hat{U}}{\partial q_{xx_i}}\right).
\end{equation}
Then
\begin{equation}
\text d\alpha = \frac{\text d^2}{\text dx^2}\left(\frac{\partial \hat{U}}{\partial q_{xx_i}}\right)\text{ d}x.
\end{equation}
Let
\begin{equation}
\text d\beta = \delta q_{x_i}\text{ d}x.
\end{equation}
Then
\begin{equation}
\beta = \int \frac{\text d}{\text dx}\delta q_i \text{ d}x = \delta q_i.
\end{equation}
Applying the integration by parts formula yields
\begin{equation}
\int_0^L\frac{\text d}{\text dx}\left(\frac{\partial \hat{U}}{\partial q_{xx_i}}\right)\delta q_{x_i}\text{ d}x = \left.\frac{\text d}{\text dx}\left(\frac{\partial \hat{U}}{\partial q_{xx_i}}\right)\delta q_i\right|_0^L - \int_0^L \frac{\text d^2}{\text dx^2}\left(\frac{\partial \hat{U}}{\partial q_{xx_i}}\right)\delta q_i\text{ d}x.
\end{equation}
Once again, the first term is a boundary condition term.
Plugging this into Eqn.~\ref{eq:strain.byparts.term1.first} yields
\begin{equation}
\int_0^L\frac{\partial \hat U}{\partial q_{xx_i}}\delta q_{xx_i}\text{ d}x = \left.\frac{\partial \hat U}{\partial q_{xx_i}}\delta q_{x_i}\right|_0^L-\left.\frac{\text d}{\text dx}\left(\frac{\partial \hat{U}}{\partial q_{xx_i}}\right)\delta q_i\right|_0^L+\int_0^L \frac{\text d^2}{\text dx^2}\left(\frac{\partial \hat{U}}{\partial q_{xx_i}}\right)\delta q_i\text{ d}x.
\label{eq:strain.byparts.term1}
\end{equation}
Similarly, we integrate the second term of Eqn.~\ref{eq:strain.byparts.initial} by parts. Let
\begin{equation}
\alpha = \frac{\partial \hat U}{\partial q_{x_i}}.
\end{equation}
Then 
\begin{equation}
\text d\alpha = \frac{\text d}{\text dx}\left(\frac{\partial \hat U}{\partial q_{x_i}}\right)\text{ d}x.
\end{equation}
Let
\begin{equation}
\text d\beta = \delta q_{x_i}\text{ d}x.
\end{equation}
Then
\begin{equation}
\beta = \int \frac{\text d}{\text dx}\delta q_i\text{ d}x = \delta q_i.
\end{equation}
Applying the integration by parts formula, we have
\begin{equation}
\int_0^L\frac{\partial \hat U}{\partial q_{x_i}}\delta q_{x_i}\text{ d}x = \left.\frac{\partial \hat U}{\partial q_{x_i}}\delta q_i\right|_0^L - \int_0^L \frac{\text d}{\text dx}\left(\frac{\partial \hat U}{\partial q_{x_i}}\right)\delta q_i\text{ d}x.
\label{eq:strain.byparts.term2}
\end{equation}
Again, the lead term is for the boundary conditions. Replacing Eqn.~\ref{eq:strain.byparts.term1} and Eqn.~\ref{eq:strain.byparts.term2} into Eqn.~\ref{eq:strain.byparts.initial} yields
\begin{equation}
\int_{t_0}^{t_1}\int_0^L\delta \hat U(q_{xx_i},q_{x_i})\text{ d}x\text{ d}t = \int_{t_0}^{t_1} \int_0^L \left\lbrace\frac{\text d^2}{\text dx^2}\left[\frac{\partial \hat{U}}{\partial q_{xx_i}}\right]-\frac{\text d}{\text dx}\left[\frac{\partial \hat U}{\partial q_{x_i}}\right]\right\rbrace\delta q_i\text{ d}x\text{ d}t + \text{B.C.'s},
\end{equation}
where the term ``B.C.'s'' is given by
\begin{equation}
\text{B.C.'s} = \left.\frac{\partial \hat U}{\partial q_{xx_i}}\delta q_{x_i}\right|_0^L-\left.\frac{\text d}{\text dx}\left(\frac{\partial \hat{U}}{\partial q_{xx_i}}\right)\delta q_i\right|_0^L + \left.\frac{\partial \hat U}{\partial q_{x_i}}\delta q_i\right|_0^L.
\end{equation}
Rewriting $\hat U$ in terms of $U$ yields
\begin{equation}
\int_{t_0}^{t_1}\delta U(q_{xx_i},q_{x_i})\text{ d}t = \int_{t_0}^{t_1} \left\lbrace\frac{\text d^2}{\text dx^2}\left[\frac{\partial U}{\partial q_{xx_i}}\right]-\frac{\text d}{\text dx}\left[\frac{\partial  U}{\partial q_{x_i}}\right]\right\rbrace\delta q_i\text{ d}t + \text{B.C.'s},
\end{equation}
with ``B.C.'s'' given by
\begin{equation}
\text{B.C.'s} = \left.\frac{\partial U}{\partial q_{xx_i}}\delta q_{x_i}\right|_0^L-\left.\frac{\text d}{\text dx}\left(\frac{\partial U}{\partial q_{xx_i}}\right)\delta q_i\right|_0^L + \left.\frac{\partial U}{\partial q_{x_i}}\delta q_i\right|_0^L.
\end{equation} 
Combining the kinetic and strain energy variations, we have
\begin{equation}
\int_{t_0}^{t_1}\delta L\text{ d}t = \int_{t_0}^{t_1}\left\lbrace-\frac{\text d}{\text dt}\left[\frac{\partial T}{\partial \dot q_i}\right]+\frac{\partial T}{\partial q_i} - \frac{\text d^2}{\text dx^2}\left[\frac{\partial U}{\partial q_{xx_i}}\right]+\frac{\text d}{\text dx}\left[\frac{\partial  U}{\partial q_{x_i}}\right]\right\rbrace\delta q_i\text{ d}t+\text{B.C.'s}.
\end{equation}

\begin{comment}


We now consider the special case of $\delta u$. In this case, we have
\begin{equation}
\int_{t_0}^{t_1}\delta L\text{ d}t = \int_{t_0}^{t_1}\left\lbrace-\frac{\text d}{\text dt}\left[\frac{\partial T}{\partial \dot u}\right]+\frac{\partial T}{\partial u} - \frac{\text d^2}{\text dx^2}\left[\frac{\partial U}{\partial u_{xx}}\right]+\frac{\text d}{\text dx}\left[\frac{\partial  U}{\partial u_x}\right]\right\rbrace\delta u\text{ d}t+\text{B.C.'s}.
\end{equation}

Considering the special circumstances involving $\delta u$, we shall now express this result for the $u$ direction by substituting in $\delta u$ from Eqn.~\ref{eq:variation.u.final}, and expressing the variation in terms of $s$.

We have
\begin{eqnarray}
\int_{t_0}^{t_1}\delta L\text{ d}t &=&
\int_{t_0}^{t_1}\left\lbrace -\frac{\text d}{\text dt}\left[\frac{\partial T}{\partial \dot s}\right]+\frac{\partial T}{\partial s} - \frac{\text d^2}{\text dx^2}\left[\frac{\partial U}{\partial s_{xx}}\right]+\frac{\text d}{\text dx}\left[\frac{\partial  U}{\partial s_x}\right]\right\rbrace {\vrule width 0in depth .1in} \\
& &\qquad \cdot\left[\delta s + \int_0^x\frac{\partial^2 v}{\partial \eta^2}\delta v\text{ d}\eta + \int_0^x\frac{\partial^2 w}{\partial \eta^2}\delta w\text{ d}\eta\right]\text{ d}t+ \text{B.C.'s}. \nonumber
\end{eqnarray}
\end{comment}

This formulation, provided the B.C.'s are properly satisfied, is to be minimized. Thus, we set
\begin{equation}
\int_{t_0}^{t_1}\delta L\text{ d}t = 0,
\end{equation}
so that, given the arbitrary nature of the test function $q_i$, it must be true that
\begin{equation}
-\frac{\text d}{\text dt}\left[\frac{\partial T}{\partial \dot q_i}\right]+\frac{\partial T}{\partial q_i} - \frac{\text d^2}{\text dx^2}\left[\frac{\partial U}{\partial q_{xx_i}}\right]+\frac{\text d}{\text dx}\left[\frac{\partial  U}{\partial q_{x_i}}\right] = 0.
\label{eq:euler.lagrange}
\end{equation}
These are known as the \emph{Euler-Lagrange Equations}. This expression describes the motion of the system.

\subsection{System Analysis}
Considering that the displacement $u$ is a function of $s,h_v,h_w$, we must analyze the special case of $\delta u$. The Chain Rule for Variations yields that given that
\begin{equation}
u = u(s,h_v,h_w) = s-(h_v+h_w),
\end{equation}
we write the variation of $u$ as
\begin{equation}
\delta u = \frac{\partial u}{\partial s}\delta s+\frac{\partial u}{\partial h_v}\delta h_v+\frac{\partial u}{\partial h_w}\delta h_w = \delta s - \delta h_v - \delta h_w.
\label{eq:variation.u}
\end{equation}
Similarly, given that
\begin{equation}
\dot{u} = \dot{u}(\dot s,\dot h_v, \dot h_w) = \dot s - (\dot h_v+\dot h_w),
\end{equation}
we write the variation of $\dot u$ as
\begin{equation}
\delta \dot u = \frac{\partial \dot u}{\partial \dot s}\delta \dot s+\frac{\partial \dot u}{\partial \dot h_v}\delta \dot h_v+\frac{\partial \dot u}{\partial \dot h_w}\delta \dot h_w = \delta \dot s - \delta \dot h_v - \delta \dot h_w.
\end{equation}
The variations $\delta h_v$, $\delta h_w$, $\delta \dot h_v$ and $\delta \dot h_w$ are not in the proper frame; that is, we need variations on $s,v,w$ only. Since $h_v,h_w$ are functions of $v,w$ respectively, we can calculate the variations as follows.
\begin{equation}
\delta h_v = \delta \frac{1}{2} \int_0^x\left(\frac{\partial v}{\partial \eta}\right)^2\text{ d}\eta.
\end{equation}
Let 
\begin{equation}
\tau = \left(\frac{\partial v}{\partial \eta}\right)^2.
\end{equation}
Then
\begin{equation}
\delta h_v = \delta \frac{1}{2} \int_0^x \tau \text{ d}\eta,
\end{equation}
and the Chain Rule for Variations yields
\begin{equation}
\delta h_v = \frac{1}{2}\int_0^x \frac{\partial \tau}{\partial v_{\eta}}\delta v_{\eta}\text{ d}\eta = \int_0^xv_{\eta}\delta v_{\eta}\text{ d}\eta,
\end{equation}
where $v_{\eta}$ is the partial derivative of $v$ with respect to $\eta$. Note that the variation is not in terms of $v$, but in terms of $v_{\eta}$; we shall leave the variation in this form for simplicity. The reason shall be clear upon application to the system.

The process is identical for $\delta h_w$, thus plugging into Eqn.~\ref{eq:variation.u} we have
\begin{equation}
\delta u = \delta s - \int_0^x \frac{\partial v}{\partial \eta}\delta v_{\eta}\text{ d}\eta - \int_0^x \frac{\partial w}{\partial \eta}\delta w_{\eta}\text{ d}\eta.
\label{eq:variation.u.final}
\end{equation}

For the terms $\delta \dot h_v$ and $\delta \dot h_w$, we consider the full terms as follows. First, we shall consider $\delta\dot h_v$.

\begin{eqnarray}
-\delta\dot h_v &=& -\delta \frac{\partial h_v}{\partial t} \\
 &=& -\frac{\partial}{\partial t}\delta h_v\\
 &=& -\frac{\partial}{\partial t}\left[-\int_0^xv_{\eta\eta}\delta v\text{ d}\eta\right] \\
 &=& \int_0^x\frac{\partial}{\partial t}\left[\frac{\partial^2v}{\partial\eta^2}\delta v\right]\text{ d}\eta\\
 &=& \int_0^x\frac{\partial^3 v}{\partial t\partial\eta^2}\delta v+\frac{\partial^2 v}{\partial \eta^2}\delta\dot v\text{ d}\eta
\end{eqnarray}
However, this is multiplied by the partial of $T$ on $\dot u$, which yields
\begin{equation}
\int_{t_0}^{t_1}\int_0^L\left(\int_0^x\frac{\partial}{\partial t}\left[\frac{\partial^2v}{\partial\eta^2}\delta v\right]\text{ d}\eta\right)\frac{\partial \hat T}{\partial \dot u}\text{ d}x\text{ d}t.
\end{equation}
We apply Lemma~\ref{lem:switch.integrals} and find
\begin{align}
 \int_{t_0}^{t_1} \int_0^L & \left(\int_L^x\frac{\partial\hat T}{\partial \dot u}\text{ d}\eta\right)\frac{\partial}{\partial t}\left[\frac{\partial^2v}{\partial x^2}\delta v\right]\text{ d}x\text{ d}t \label{eq:lemma_switch_vdot}\\
&= \int_{t_0}^{t_1}\int_0^L\left(\int_L^x\frac{\partial\hat T}{\partial \dot u}\text{ d}\eta\right)\left[\frac{\partial^3v}{\partial t\partial x^2}\delta v+\frac{\partial^2 v}{\partial x^2}\delta\dot v\right]\text{ d}x\text{ d}t. \nonumber
\end{align}
The term involving $\delta \dot v$ must be reconfigured so that the variation is in terms of $v$ instead of $\dot v$. To accomplish this, we perform integration by parts, as below.
Consider the integral
\begin{equation}
\int_{t_0}^{t_1}\frac{\partial^2 v}{\partial x^2}\delta \dot v\text{ d}t.
\end{equation}
Let
\begin{equation}
\alpha = \frac{\partial^2 v}{\partial x^2}.
\end{equation}
Then
\begin{equation}
\text d\alpha = \frac{\partial}{\partial t}\frac{\partial^2 v}{\partial x^2}\text dt = \frac{\partial^3 v}{\partial t\partial x^2}\text dt.
\end{equation}
Let
\begin{equation}
\text d\beta = \delta\dot v\text{ d}t.
\end{equation}
Then
\begin{equation}
\beta = \int \delta\frac{\partial v}{\partial t}\text dt = \int\frac{\partial}{\partial t} \delta v\text{ d}t = \delta v.
\end{equation}
Applying the integration by parts formula, we have
\begin{equation}
\int_{t_0}^{t_1}\frac{\partial^2 v}{\partial x^2}\delta \dot v\text{ d}t = \left.\frac{\partial^2v}{\partial x^2}\delta v\right|_{t_0}^{t_1} - \int_{t_0}^{t_1}\frac{\partial^3 v}{\partial t\partial x^2}\delta v\text{ d}t,
\end{equation}
and the first term on the right side vanishes by definition of $\delta$, so we have that 
\begin{equation}
\int_{t_0}^{t_1}\frac{\partial^2 v}{\partial x^2}\delta \dot v\text{ d}t =- \int_{t_0}^{t_1}\frac{\partial^3 v}{\partial t\partial x^2}\delta v\text{ d}t,
\end{equation}
and clearly
\begin{equation}
\frac{\partial^2 v}{\partial x^2}\delta \dot v=-\frac{\partial^3 v}{\partial t\partial x^2}\delta v.
\end{equation}
Substituting this result in Eqn.~\ref{eq:lemma_switch_vdot} yields
\begin{equation}
\int_{t_0}^{t_1}\int_0^L\left(\int_L^x\frac{\partial\hat T}{\partial \dot u}\text{ d}\eta\right)\left[\frac{\partial^3v}{\partial t\partial x^2}\delta v-\frac{\partial^3v}{\partial t\partial x^2}\delta v\right]\text{ d}x\text{ d}t = 0.
\end{equation}
The result is precisely the same for $\delta\dot h_w$, so that 
\begin{equation}
\delta\dot u = \delta\dot s.
\end{equation}

In our analysis we may now substitute these results for $\delta u,\delta\dot u$. We shall now evaluate the terms $\frac{\partial \hat T}{\partial u}$ and $\frac{\partial \hat T}{\partial \dot u}$.

\begin{equation}
\frac{\partial\hat T}{\partial u} = \frac{\partial \hat T}{\partial s}\frac{\partial s}{\partial u}+\frac{\partial \hat T}{\partial h_v}\frac{\partial h_v}{\partial u}+\frac{\partial \hat T}{\partial h_w}\frac{\partial h_w}{\partial u}.
\end{equation}
Since
\begin{equation}
u = s-(h_v+h_w),
\end{equation}
we have
\begin{eqnarray}
s &= u+(h_v+h_w),\\
h_v &= s-(u+h_w),\\
h_w &= s-(u+h_v),
\end{eqnarray}
so that
\begin{eqnarray}
\frac{\partial s}{\partial u} &= 1,\\
\frac{\partial h_v}{\partial u} &= -1,\\
\frac{\partial h_w}{\partial u} &= -1,
\end{eqnarray}
which yields
\begin{equation}
\frac{\partial\hat T}{\partial u} = \frac{\partial \hat T}{\partial s}-\left(\frac{\partial \hat T}{\partial h_v}+\frac{\partial \hat T}{\partial h_w}\right).
\end{equation}
Similarly,
\begin{equation}
\frac{\partial\hat T}{\partial \dot u} = \frac{\partial \hat T}{\partial \dot s}-\left(\frac{\partial \hat T}{\partial \dot h_v}+\frac{\partial \hat T}{\partial \dot h_w}\right).
\end{equation}
Given the linearity of both $\frac{\partial \hat T}{\partial u}$ and $\frac{\partial \hat T}{\partial \dot u}$, we may calculate these in terms of $u,\dot u$ and substitute for $u=s-(h_v+h_w)$ upon conclusion of the calculation.

We begin the analysis with the first term of Eqn.~\ref{eq:euler.lagrange}.
\begin{eqnarray}
\left.-\frac{\text d}{\text dt}\left[\frac{\partial T}{\partial \dot q_i}\right]\right|_{q_i=u} 
&=& -\left(\frac{\rho A}{2}\right)\frac{\text d}{\text dt}\int_0^L[2\dot u-2\Omega v]\text{ d}x,\\
&=& -\rho A\int_0^L [\ddot u-\dot \Omega v-\Omega \dot v]\text{ d}x,
\end{eqnarray}

Continuing, we have
\begin{eqnarray}
\left.-\frac{\text d}{\text dt}\left[\frac{\partial T}{\partial \dot q_i}\right]\right|_{q_i=v}
&=& -\left(\frac{\rho A}{2}\right)\frac{\text d}{\text dt}\int_0^L[2\dot v+2\Omega (a+x+u)]\text{ d}x,\\
&=& -\int_0^L \rho A [\ddot v + \dot \Omega (a+x+u)+\Omega\dot u]\text{ d}x.
\end{eqnarray}
In $w$,
\begin{eqnarray}
\left.-\frac{\text d}{\text dt}\left[\frac{\partial T}{\partial \dot q_i}\right]\right|_{q_i=w}
&=& -\left(\frac{\rho A}{2}\right)\frac{\text d}{\text dt}\int_0^L\left[2\dot w \right]\text{ d}x,\\
&=& -\int_0^L \rho A [\ddot w]\text{ d}x.
\end{eqnarray}
Next we shall analyze the second term. In $u$,
\begin{eqnarray}
\left.\frac{\partial T}{\partial q_i}\right|_{q_i=u} 
&=& \frac{\rho A}{2}\int_0^L 2\dot v\Omega+2\Omega^2(u+a+x)\text{ d}x,\\
&=& \rho A\int_0^L  [\dot v\Omega+\Omega^2(u+a+x)]\text{ d}x.
\end{eqnarray}
In $v$,
\begin{eqnarray}
\left.\frac{\partial T}{\partial q_i}\right|_{q_i=v} 
&=& \frac{\rho A}{2}\int_0^L [-2\Omega\dot u+2\Omega^2v ]\text{ d}x,\\
&=& \rho A\int_0^L  [\Omega^2v-\Omega\dot u] \text{ d}x.
\end{eqnarray}
Finally, in $w$,
\begin{eqnarray}
\left.\frac{\partial T}{\partial q_i}\right|_{q_i=w} 
&=& 0
\end{eqnarray}
The third and fourth terms are from strain energy. The third term is analyzed as follows. In $s$,
\begin{eqnarray}
\left.-\frac{\text d^2}{\text dx^2}\left[\frac{\partial U}{\partial q_{xx_i}}\right]\right|_{q_i=s} 
&=& -\frac{\text d^2}{\text dx^2}\left[ 0\right],\\
&=& 0.
\end{eqnarray}
In $v$,
\begin{eqnarray}
\left.-\frac{\text d^2}{\text dx^2}\left[\frac{\partial U}{\partial q_{xx_i}}\right]\right|_{q_i=v} 
&=& -\frac{\text d^2}{\text dx^2}\int_0^L\frac{EI_{zz}}{2}\left[ 2v_{xx} \right]\text{ d}x,\\
&=& -E\int_0^L [I_{zz}]_{xx}[v_{xx}]+2[I_{zz}]_{x}[v_{xxx}]+\\
& & \qquad [I_{zz}][v_{xxxx}]\text{ d}x. \nonumber
\end{eqnarray}
We have already assumed constant cross-section area $A$, and we shall also consider that moment of inertia is constant, so that $I_{qq_{x}}=0$, $qq\in\{yy,zz\}$. Thus we write
\begin{eqnarray}
\left.-\frac{\text d^2}{\text dx^2}\left[\frac{\partial U}{\partial q_{xx_i}}\right]\right|_{q_i=v} 
&=& -EI_{zz}\int_0^L v_{xxxx}\text{ d}x. 
\end{eqnarray}
In $w$,
\begin{eqnarray}
\left.-\frac{\text d^2}{\text dx^2}\left[\frac{\partial U}{\partial q_{xx_i}}\right]\right|_{q_i=w} 
&=& -\frac{\text d^2}{\text dx^2}\int_0^L\frac{EI_{yy}}{2}\left[ 2w_{xx} \right]\text{ d}x\\
&=& -EI_{yy}\int_0^L w_{xxxx}\text{ d}x. 
\end{eqnarray}
Finally, the fourth term is given as follows. In $s$,
\begin{eqnarray}
\left.\frac{\text d}{\text dx}\left[\frac{\partial U}{\partial q_{x_i}}\right]\right|_{q_i=s} 
&=& \frac{\text d}{\text dx}\int_0^L\frac{EA}{2}\left[ 2s_{x} \right]\text{ d}x,\\
&=& EA\int_0^L s_{xx}\text{ d}x.
\end{eqnarray}
In $v$,
\begin{eqnarray}
\left.\frac{\text d}{\text dx}\left[\frac{\partial U}{\partial q_{x_i}}\right]\right|_{q_i=v} 
&=& \frac{\text d}{\text dx}\int_0^L\frac{EA}{2}\left[ 0\right]\text{ d}x\\
&=& 0.
\end{eqnarray}
\begin{eqnarray}
\left.\frac{\text d}{\text dx}\left[\frac{\partial U}{\partial q_{x_i}}\right]\right|_{q_i=w} 
&=& \frac{\text d}{\text dx}\int_0^L\frac{EA}{2}\left[ 0\right]\text{ d}x\\
&=& 0.
\end{eqnarray}

It is important to note that the equations are broken up into their respective coordinates purposefully. One reason is to illuminate the zero evaluations. The other is that the Euler-Lagrange equations must be evaluated for each coordinate, $s,v,w$. We will summarize these results by grouping the non-zero terms into their respective coordinate references.

We first consider the Lagrangian for the various coordinates. We use the linearity to write the derivatives of the potential function $T$ in terms of $u$. Furthermore, we show the Lagrangian in terms of the three coordinates $s,v,w$ as separate equations. The expanded Lagrangian under variation is written as
\begin{eqnarray}
\left.\delta\int_{t_0}^{t_1}L\text{ d}t\right|_s = \int_{t_0}^{t_1}\left\lbrace-\frac{\text d}{\text dt}\left[\frac{\partial T}{\partial \dot u}\right]+\frac{\partial T}{\partial u} - \frac{\text d^2}{\text dx^2}\left[\frac{\partial U}{\partial s_{xx}}\right]+\frac{\text d}{\text dx}\left[\frac{\partial  U}{\partial s_{x}}\right]\right\rbrace\delta s\text{ d}t,\\
\left.\delta\int_{t_0}^{t_1}L\text{ d}t\right|_v =\int_{t_0}^{t_1}\left\lbrace-\frac{\text d}{\text dt}\left[\frac{\partial T}{\partial \dot v}\right]+\frac{\partial T}{\partial v} - \frac{\text d^2}{\text dx^2}\left[\frac{\partial U}{\partial v_{xx}}\right]+\frac{\text d}{\text dx}\left[\frac{\partial  U}{\partial v_{x}}\right]\right\rbrace\delta v\text{ d}t\\
-\int_{t_0}^{t_1}\frac{\partial T}{\partial u}\left[\int_0^xv_{\eta}\delta v_{\eta}\text{ d}\eta\right]\text{ d}t,\nonumber\\
\left.\delta\int_{t_0}^{t_1}L\text{ d}t\right|_w =\int_{t_0}^{t_1}\left\lbrace-\frac{\text d}{\text dt}\left[\frac{\partial T}{\partial \dot w}\right]+\frac{\partial T}{\partial w} - \frac{\text d^2}{\text dx^2}\left[\frac{\partial U}{\partial w_{xx}}\right]+\frac{\text d}{\text dx}\left[\frac{\partial  U}{\partial w_{x}}\right]\right\rbrace\delta w\text{ d}t\\
-\int_{t_0}^{t_1}\frac{\partial T}{\partial u}\left[\int_0^xw_{\eta}\delta w_{\eta}\text{ d}\eta\right]\text{ d}t,\nonumber
\end{eqnarray}
where we have appended the terms for $\delta h_v,\delta h_w$ into the $v$ and $w$ variation directions.

Now we write down the $s$ coordinate equations, after plugging in for $u = s-(h_v+h_w)$.
\begin{eqnarray}
& & \int_{t_0}^{t_1}\left[-\rho A\int_0^L  \left[\ddot s-\frac{\partial}{\partial t}(\dot h_v+\dot h_w)-\dot \Omega v-\Omega \dot v\right]\text{ d}x\right. \nonumber \\
& & \qquad +\rho A\int_0^L  [\lbrace s-(h_v+h_w)+a+x\rbrace\Omega^2+\dot v\Omega]\text{ d}x \\
& & \qquad \qquad \left.+EA\int_0^L s_{xx}\text{ d}x \right]\delta s\text{ d}t\nonumber
\end{eqnarray}
Applying the variation, canceling integrals and simplifying yields the following.
\begin{equation}
\rho A \left[\ddot s-\frac{\partial}{\partial t}(\dot h_v+\dot h_w)-\dot \Omega v+2\Omega\dot v-\lbrace s-(h_v+h_w)+a+x\rbrace\Omega^2\right] = EAs_{xx}.
\end{equation}
Next, we write the $v$ coordinate equations after substituting for $u$.
\begin{eqnarray}
& & \int_{t_0}^{t_1}\left[-\rho A \int_0^L [\ddot v + \dot \Omega (a+x+s-(h_v+h_w))+\Omega(\dot s - (\dot h_v+\dot h_w))]\text{ d}x\right.\nonumber \\ 
& & \qquad\left. +\rho A\int_0^L  [-\Omega(\dot s-(\dot h_v+\dot h_w))+\Omega^2v] \text{ d}x + EI_{zz}\int_0^L v_{xxxx}\text{ d}x\right]\delta v\text{ d}t\\
& & \qquad -\int_{t_0}^{t_1}\rho A\int_0^L [\lbrace s-(h_v+h_w)+a+x\rbrace\Omega^2+\dot v\Omega]\left[\int_0^xv_{\eta}\delta v_{\eta}\text{ d}\eta\right]\text{ d}x\text{ d}t\nonumber
\end{eqnarray}
We cannot apply the variation unless the variation is in the proper domain. We transform the variation to be in terms of $v$ using integration by parts. To solve this, we first apply Lemma~\ref{lem:switch.integrals}, which yields
\begin{eqnarray}
& & \int_{t_0}^{t_1}\left[-\rho A \int_0^L [\ddot v + \dot \Omega (a+x+s-(h_v+h_w))+\Omega(\dot s - (\dot h_v+\dot h_w))]\text{ d}x\right.\nonumber \\ 
& & \qquad +\rho A\int_0^L  [-\Omega(\dot s-(\dot h_v+\dot h_w))+\Omega^2v] \text{ d}x + EI_{zz}\int_0^L v_{xxxx}\text{ d}x\\
& & \qquad -\left.\rho A\int_0^L \left[\int_L^x\lbrace s-(h_v+h_w)+a+\eta\rbrace\Omega^2+\dot v\Omega\text{ d}\eta\right] v_{x}\delta v_x\text{ d}x\right]\text{ d}t.\nonumber
\end{eqnarray}
We now apply integration by parts on the last term. Let 
\begin{equation}
\alpha = \left[\int_L^x\lbrace s-(h_v+h_w)+a+\eta\rbrace\Omega^2+\dot v\Omega\text{ d}\eta\right] v_{x},
\end{equation}
then
\begin{equation}
\text{ d}\alpha = \frac{\partial }{\partial x}\left\lbrace\left[\int_L^x( s-(h_v+h_w)+a+\eta)\Omega^2+\dot v\Omega\text{ d}\eta\right] v_{x}\right\rbrace.
\end{equation}
Let 
\begin{equation}
\text{ d}\beta = \delta v_x\text{ d}x,
\end{equation}
then
\begin{equation}
\beta = \delta v.
\end{equation}
Application of the integration by parts formula yields
\begin{eqnarray}
& &\int_0^L \left[\int_L^x(s-(h_v+h_w)+a+\eta)\Omega^2+\dot v\Omega\text{ d}\eta\right] v_{x}\delta v_x\text{ d}x \\
&=& \left.\vphantom{\frac{0}{0}}\alpha\delta v\right|_0^L-\int_0^L\frac{\partial }{\partial x}\left\lbrace\left[\int_L^x( s-(h_v+h_w)+a+\eta)\Omega^2+\dot v\Omega\text{ d}\eta\right] v_{x}\right\rbrace\delta v\text{ d}x,\nonumber
\end{eqnarray}
where the term $\alpha\delta v$ vanishes on the boundary by definition of $\delta$.
Applying the variation, canceling integrals and simplifying yields the following.
\begin{eqnarray}
& & -\rho A [\ddot v + \dot \Omega (a+x+s-(h_v+h_w))+2\Omega(\dot s - (\dot h_v+\dot h_w))-\Omega^2v] \\ 
& & \qquad +\rho A \frac{\partial }{\partial x}\left\lbrace\left[\int_L^x( s-(h_v+h_w)+a+\eta)\Omega^2+\dot v\Omega\text{ d}\eta\right] v_{x}\right\rbrace = -EI_{zz}v_{xxxx}\nonumber
\end{eqnarray}
Lastly, we write the $w$ coordinate equations.
\begin{eqnarray}
& & \int_{t_0}^{t_1}\left[-\rho A\int_0^L \ddot w\text{ d}x+EI_{yy}\int_0^L w_{xxxx}\text{ d}x\right]\delta w\text{ d}t \\
& & \qquad +\int_{t_0}^{t_1}\rho A\int_0^L  [( s-(h_v+h_w)+a+x)\Omega^2+\dot v\Omega]\left[\int_0^xw_{\eta}\delta w_{\eta}\text{ d}\eta\right]\text{ d}x\text{ d}t\nonumber 
\end{eqnarray}
Again, we must apply Lemma~\ref{lem:switch.integrals} to move the variation.
\begin{eqnarray}
& & \int_{t_0}^{t_1}\left[-\rho A\int_0^L \ddot w\text{ d}x+EI_{yy}\int_0^L w_{xxxx}\text{ d}x\right]\delta w\text{ d}t \\
& & \qquad +\int_{t_0}^{t_1}\rho A\int_0^L \int_L^x[(s-(h_v+h_w)+a+\eta)\Omega^2+\dot v\Omega]\text{ d}\eta w_{x}\delta w_x\text{ d}x\text{ d}t\nonumber 
\end{eqnarray}
We perform integration by parts in the same way as in the case of $v$, which yields
\begin{eqnarray}
& &\int_0^L \left[\int_L^x(s-(h_v+h_w)+a+\eta)\Omega^2+\dot v\Omega\text{ d}\eta\right] w_{x}\delta w_x\text{ d}x \\
&=& \left.\vphantom{\frac{0}{0}}\alpha\delta w\right|_0^L-\int_0^L\frac{\partial }{\partial x}\left\lbrace\left[\int_L^x( s-(h_v+h_w)+a+\eta)\Omega^2+\dot v\Omega\text{ d}\eta\right] w_{x}\right\rbrace\delta w\text{ d}x,\nonumber
\end{eqnarray}
where the term $\alpha\delta w$ vanishes on the boundary by definition of $\delta$.
Applying the variation, canceling integrals and simplifying yields the following.
\begin{equation}
-\rho A \ddot w+\rho A \frac{\partial }{\partial x}\left\lbrace\left[\int_L^x( s-(h_v+h_w)+a+\eta)\Omega^2+\dot v\Omega\text{ d}\eta\right] w_{x}\right\rbrace = -EI_{yy} w_{xxxx}
\end{equation}
Note that the product of $s,h_v,h_w,$ and $\dot v$ with either $v_{x}$ or $w_{x}$ yields nonlinear terms. Furthermore, recall that $h_v,h_w$ and their derivatives are nonlinear. We shall exclude these terms. Additionally, we shall \emph{ad hoc} add the forcing terms $p_v,p_w$. This simplification yields the following system of linear differential equations. This is an approximation to the equations of motion of a rotating cantilever beam.
\begin{equation}
\rho A \left[\ddot s-\dot \Omega v+2\Omega\dot v-(s+a+x)\Omega^2\right] = EAs_{xx}.
\label{eq:EL_s_final}
\end{equation}
\begin{equation}
-\rho A [\ddot v + \dot \Omega (s+a+x)+2\Omega\dot s-\Omega^2v]+\rho A\frac{\partial}{\partial x}\left\lbrace \int_L^x\Omega^2(a+\eta)\text{ d}\eta v_{x}\right\rbrace = p_v-EI_{zz}v_{xxxx} 
\label{eq:EL_v_final}
\end{equation}
\begin{equation}
-\rho A \ddot w+\rho A \frac{\partial}{\partial x}\left\lbrace\int_L^x\Omega^2(a+\eta)\text{ d}\eta w_{x}\right\rbrace = p_w-EI_{yy} w_{xxxx}
\label{eq:EL_w_final}
\end{equation}

\subsection{Discretizing the System}
\label{subsec:discretizing}
We now have the final equations of motion, with the caveat that `suitable' test functions $q_i$ must be selected. `Suitable' means that the functions are in $H^2$ in the interval $[0,L]$, and that all boundary and initial conditions are satisfied by the functions. There are standard test functions that are frequently used; choosing which class of test functions to use is dependent upon the physics of the problem. Furthermore, the test functions should match the behavior of the spatial displacements of the system. Polynomial shapes are reasonable approximations to how we expect the beam to displace from bending and stretching, so choosing polynomials for test functions is logical.

%One such example are the Hermite polynomials (Ref.~\cite{kwon2000finite}). Hermite polynomials are defined in all of $\mathbb R$, are infinitely smooth, and can be constructed to satisfy any requisite boundary conditions.

In order that we might find suitable interpolating polynomials for the discretized system, we begin by considering a rod consisting of one element and two nodes.

\begin{figure}[ht!]
\caption{Single Element, Two Node Beam}
\centering
\includegraphics[width=0.35\textwidth]{images/2-node-rod.eps}
\end{figure}

The displacement directions are $s,v,w$, corresponding to stretch, chordwise and flapwise bending respectively. Stretch is uni-axial and thus only has one degree of freedom, which is from displacement. Bending, however, has both displacement and bending angle, so the flapwise and chordwise displacements each have two degrees of freedom: one from displacement, the other from bending angle.

The stretch degrees of freedom applied to the two node single element beam yields one degree of freedom at two distinct nodes for a total of two degrees of freedom. The smallest polynomial which may represent this system has two coefficients and is therefore linear.

\begin{figure}[ht!]
\caption{Single Element, Two Node Stretch Displacement Diagram}
\centering
\includegraphics[width=0.35\textwidth]{images/stretch_2node.eps}
\label{fig:2-node-stretch-dof}
\end{figure}

Considering the $n$-th two node element along the rod, we represent the stretch degrees of freedom (DOF) as $s_n,s_{n+1}$ as shown in Fig.~\ref{fig:2-node-stretch-dof}. The interpolating stretch polynomial can be expressed as 

\begin{equation}
s(x) = c_0 + c_1x,
\label{eq:stretch.polynomial}
\end{equation}
or, in vector form,
\begin{equation}
s(x) = 
\begin{pmatrix}
1 & x
\end{pmatrix}
\begin{pmatrix}
c_0 \\
c_1
\end{pmatrix}.
\end{equation}
In this form, we express the polynomial vector $\vec p$ and coefficient vector $\vec c$ as follows.
\begin{eqnarray}
\vec{p} =
\begin{pmatrix}
1 \\
x
\end{pmatrix}, \\
\vec{c} = 
\begin{pmatrix}
c_0 \\
c_1
\end{pmatrix},
\end{eqnarray}
so that the polynomial can be written as
\begin{equation}
s(x) = \vec{p}^\top\vec{c}.
\label{eq:stretch.polynomial.vector}
\end{equation}
Plugging in constraints on the rod so that $x_n=0$ and $x_{n+1}=L$, we can write the polynomial from Eqn.~\ref{eq:stretch.polynomial} as
\begin{eqnarray}
s(0) = 
\begin{pmatrix}
1 & 0
\end{pmatrix}
\begin{pmatrix}
c_0 \\
c_1
\end{pmatrix}
= c_0 = s_0,\\
s(L) = 
\begin{pmatrix}
1 & L
\end{pmatrix}
\begin{pmatrix}
c_0 \\
c_1
\end{pmatrix}
= c_0 + c_1L = s_1.
\end{eqnarray}
We then write the element vector $\vec s$ as
\begin{equation}
\vec s = 
\begin{pmatrix}
s_0 \\
s_1
\end{pmatrix},
\end{equation}
and the matrix $\mathbf{A}$ as
\begin{equation}
\mathbf A = 
\begin{pmatrix}
1 & 0 \\
1 & L
\end{pmatrix},
\end{equation}
so that we can write the system as
\begin{equation}
\vec{s} = \mathbf{A}\vec{c}.
\end{equation}
Solving for the coefficients $\vec{c}$ yields
\begin{equation}
\vec{c} = \mathbf{A}^{-1}\vec{s} = 
\begin{pmatrix}
1 & 0 \\
-1/L & 1/L
\end{pmatrix}
\begin{pmatrix}
s_0 \\
s_1
\end{pmatrix}.
\label{eq:stretch.inverse.coeffs}
\end{equation}
Plugging the $\vec{c}$ coefficient vector from Eqn.~\ref{eq:stretch.inverse.coeffs} into Eqn.~\ref{eq:stretch.polynomial.vector} yields
\begin{equation}
s(x) = \vec{p}^\top\mathbf{A}^{-1}\vec{s}.
\end{equation}
The product $\vec{p}^\top\mathbf{A}^{-1}$ yields the following vector
\begin{equation}
\vec{H}^\top = \vec{p}^\top\mathbf{A}^{-1} = 
\begin{pmatrix}
1 & x
\end{pmatrix}
\begin{pmatrix}
1 & 0 \\
-1/L & 1/L
\end{pmatrix} = 
\begin{pmatrix}
1-x/L & x/L
\end{pmatrix},
\end{equation}
so that
\begin{equation}
\vec H(x) =
\begin{pmatrix}
1-x/L \\
x/L
\end{pmatrix}.
\end{equation}
The first entry in $\vec H$, say $H_1(x)$, is a polynomial in $x$. This polynomial is a scaled first order Hermitian polynomial. The scaling is from $L$, the arbitrary length. If $L=1$, this is precisely the Hermitian polynomial of order one. Together with the second term, they form the Hermitian interpolating basis polynomials for the stretch component of our physical system.

The product $\vec{H}^\top\vec{s}$ yields the system stretch polynomial in terms of the interpolating polynomials and the stretch displacements at the nodal positions. When at the first node, $H_2 = 0$ so that the stretch displacement $s$ is precisely equal to the measured stretch at the first node; similarly, when at the second node, $H_1 = 0$ so that the stretch displacement $s$ is precisely equal to the measured stretch at the second node. Elsewhere, the stretch is a linear combination of the measured stretch terms.

In this case, we chose the terminal position to be $L$, the total length of the rod. However, this formulation works at the $n$-th element along the rod, where $L$ is replaced by the distance $l$ between the $n$-th and $(n+1)$-th nodes. For the general element $n$, the starting point is still $0$, since we may index each element and treat it as a two node single element rod. Thus all nodes are accounted for and the entire rod can be interpolated by the elemental polynomial functions. This provides a discretization for the entire rod and we can represent the stretch displacement with a matrix of size $(n+1) \times (n+1)$ for $n$ elements.

In a similar fashion, we now describe the bending displacements $v,w$. For each of these displacements, there is also a bending angle. Thus, for each node there are two degrees of freedom, and for a single element two node beam there will be four degrees of freedom. The smallest interpolatory polynomial that can account for four DOF has degree three. 

\begin{figure}[ht!]
\caption{Single Element, Two Node Bending Displacement Diagram}
\centering
\includegraphics[width=0.35\textwidth]{images/beam_dof.png}
\label{fig:2-node-bending-dof}
\end{figure}

Considering the $n$-th two node element along the rod, we represent the bending DOF as $y_n,\theta_n,y_{n+1},\theta_{n+1}$ as shown in Fig.~\ref{fig:2-node-bending-dof}. Here, bending is represented by $y$ as a generic coordinate frame to represent either $v$ or $w$ bending. We shall plug $v,w$ in for $y$ at the conclusion of the calculation. 

The interpolating bending polynomial can be expressed as 
\begin{equation}
y(x) = c_0+c_1x+c_2x^2+c_3x^3.
\label{eq:bending.poly}
\end{equation}
The coefficient vector $\vec c$ and polynomial vector $\vec p$ are expressed as
\begin{eqnarray}
\vec{c} = 
\begin{pmatrix}
c_0 \\
c_1 \\
c_2 \\
c_3
\end{pmatrix}, \\
\vec{p} = 
\begin{pmatrix}
1 \\
x \\
x^2 \\
x^3
\end{pmatrix}.
\end{eqnarray}
The bending angle $\theta_i$ at each node can be locally approximated if the bending angle is small. Large bending angles would certainly permanently deform or break the beam, so this assumption is valid relative to the physical system. Thus, the bending approximation is given to be equal to the slope at the node, or
\begin{equation}
\theta(x) \approx \frac{\text dy}{\text dx} = c_1+2c_2x+3c_3x^2
\label{eq:bending.theta.poly}
\end{equation}

Following a similar path to that taken in the stretch interpolation, we plug in $x=0$ and $x=L$, solve the system for the coefficient vector $\vec c$, and back substitute to find the Hermitian polynomial vector $\vec H(x)$.

The vector form of the bending function is expressed as
\begin{equation}
y(x) = \vec{p}^\top\vec{c}.
\label{eq:bending.vec}
\end{equation}
Plugging in the limits $0,L$ into Eqn.~\ref{eq:bending.poly} and Eqn.~\ref{eq:bending.theta.poly}, we have
\begin{eqnarray}
y(0) = c_0 = y_1\\
\theta(0) = c_1 = \theta_1 \\
y(L) = c_0+c_1L+c_2L^2+c_3L^3 = y_2 \\
\theta(L) = c_1+2c_2L+3c_2L^2 = \theta_2
\end{eqnarray}
We label the vector $\vec{y}$ as
\begin{equation}
\vec{y} = 
\begin{pmatrix}
y_1 \\
\theta_1 \\
y_2 \\
\theta_2
\end{pmatrix}
\label{eq:bending.vector.y}
\end{equation}
and the matrix $\mathbf A$ as
\begin{equation}
\mathbf{A} = 
\begin{pmatrix}
1 & 0 & 0 & 0 \\
0 & 1 & 0 & 0 \\
1 & L & L^2 & L^3 \\
0 & 1 & 2L & 3L^2
\end{pmatrix}.
\end{equation}
We now have the system
\begin{equation}
\vec{y} = \mathbf{A}\vec{c},
\end{equation}
and we can find the coefficients $\vec{c}$ as
\begin{equation}
\vec{c} = \mathbf{A}^{-1}\vec{y},
\end{equation}
where
\begin{equation}
\mathbf{A}^{-1} = 
\begin{pmatrix}
1 & 0 & 0 & 0 \\
0 & 1 & 0 & 0 \\
-\frac{3}{L^2} & -\frac{2}{L} & \frac{3}{L^2} & -\frac{1}{L} \\
\frac{2}{L^3} & \frac{1}{L^2} & -\frac{2}{L^3} & \frac{1}{L^2} \\
\end{pmatrix}
\end{equation}
Plugging this back into Eqn.~\ref{eq:bending.vec}, we have
\begin{equation}
y(x) = \vec{p}^\top\mathbf{A}^{-1}\vec{y}.
\end{equation}
The product $\vec{p}^\top\mathbf{A}^{-1}$ yields the Hermite interpolating polynomials, given as
\begin{equation}
\vec{H} = 
\begin{pmatrix}
1 - \frac{3x^2}{L^2} + \frac{2x^3}{L^3} \\
x - \frac{2x^2}{L} + \frac{x^3}{L^2} \\
\frac{3x^2}{L^2} - \frac{2x^3}{L^3} \\
-\frac{x^2}{L} + \frac{x^3}{L^2}
\end{pmatrix}.
\label{eq:bending.hermitian.vector}
\end{equation}
Once again, if we consider the discretized rod of $n$ elements, each element can be treated as a single element two node rod, and this interpolation will satisfactorily approximate the beam deflection between the nodes. 

We now have elemental representations of stretch and bending corresponding to the displacements $s,v,w$. What remains is to describe how the elements fit into the system of equations we found.

To be clear, we shall state the requirements of the approximation technique we shall employ. Firstly, consider the results from Appendix~\ref{app:variations}. We have a system of equations derived through variations, and is thusly in variational form, so the Ritz approximation technique may be employed. However, we add in \emph{ad hoc} non-conservative terms to the governing equations once the variational form is procured. Since the new equations did \emph{not} come from applying variations, we are no longer guaranteed that Ritz approximations will properly emulate the system. Thus, we must employ Ritz-Galerkin (or Galerkin-Petrov) weighted residual minimization. This technique uses the approximating test functions as the weighting functions. In order to employ this technique, it is prudent to ensure that our test functions satisfy the requirements of the test functions as well as the weighting functions.

The test functions must satisfy the following criteria.
For test functions of the form
\begin{eqnarray}
s(x) &\approx& \sum_{i=1}^nc_i^s\phi_i^u(x) + \phi_0^u(x) \\
v(x) &\approx& \sum_{i=1}^nc_i^v\phi_i^v(x) + \phi_0^v(x) \\
w(x) &\approx& \sum_{i=1}^nc_i^w\phi_i^w(x) + \phi_0^w(x),
\end{eqnarray}
\begin{itemize}
\item $\phi_0^\alpha \text{ } (\alpha = s,v,w)$ should satisfy the specified essential boundary conditions corresponding to their displacement direction
\item $\phi_i^\alpha \text{ } (\alpha = s,v,w; i = 1,\dots,n)$ should satisfy 
\begin{itemize}
\item continuity in accordance with the variational method
\item the homogeneous form of the specified essential boundary conditions
\item linear independence and completeness.
\end{itemize}
\end{itemize}

Our test functions shall be Hermite polynomials of order 1 or greater. The Hermite polynomials are continuous on the boundary, complete, and linearly independent. The essential boundary conditions have yet to be described. We will address the essential boundary condition requirements after applying the approximating functions to the system.

%The discussion begins by analyzing the discretized energy equations. In order that we might tie this discussion to the commonly used methods in literature, some terminology is required. Let 
%\begin{equation}
%a_n(x,\alpha_1,\alpha_2,\dots,\alpha_n) = \sum_{i=1}^n{\alpha_iB_i(x)}.
%\end{equation}
%Consider $f = a_n^2$. Then
%\begin{equation}
%\frac{\partial f}{\partial \alpha_j} = \frac{\partial f}{\partial a_n}\frac{\partial a_n}{\partial \alpha_j} = 2a_nB_j(x) = 2\sum_{i=1}^{n}\alpha_i B_i(x)B_j(x).
%\end{equation}
%Now, $\frac{\partial f}{\partial \alpha_j}$, $j=1,\dots,n$ generates a system in $\alpha_i$, $i=1,\dots,n$, which can be expressed in a matrix formed of elemental entries
%\begin{equation}
%F_{ij} = 2 B_i(x) B_j(x),
%\end{equation}
%multiplied by the system vector $\vec{\alpha}$ as
%\begin{equation}
%\vec{\alpha} = 
%\begin{pmatrix}
%\alpha_1 \\ 
%\alpha_2 \\ 
%\vdots \\ 
%\alpha_n
%\end{pmatrix}.
%\end{equation}
%Observe that for the notation
%\begin{equation}
%\vec B = 
%\begin{pmatrix}
%B_1(x) \\ 
%B_2(x) \\ 
%\vdots \\ 
%B_n(x)
%\end{pmatrix},
%\end{equation}
%we may write the matrix $\mathbf{F}$ as
%\begin{equation}
%\mathbf{F} = 2\vec{B}\vec{B}^\top.
%\end{equation}
%This is the common notation for Finite Element matrices, and shall be employed in the discretized energy equations.

The bending element displacement and acceleration vectors from Eqn.~\ref{eq:bending.vector.y} can be expressed as
\begin{equation}
\vec y = 
\begin{pmatrix}
y_i \\
\theta_i \\
y_{i+1} \\
\theta_{i+1}
\end{pmatrix},
\end{equation}
\begin{equation}
\ddot{\vec y} = 
\begin{pmatrix}
\ddot y_i \\
\ddot\theta_i \\
\ddot y_{i+1} \\
\ddot\theta_{i+1}
\end{pmatrix},
\end{equation}
where $y_i$ represents the displacement at node $i$, and $\theta_i$ represents the bending angle at node $i$. Similarly, we express the Hermitian vector from Eqn.~\ref{eq:bending.hermitian.vector} as
\begin{equation}
\vec{H} = 
\begin{pmatrix}
H_1(x) \\
H_2(x) \\
H_3(x) \\
H_4(x)
\end{pmatrix}.
\end{equation}
We write the function $y(x)$ from Eqn.~\ref{eq:bending.poly} in terms of the vectors $\vec{y}$ and $\vec H$:
\begin{equation}
y(x) = \sum_{j=1}^4 H_j(x)y_j = \vec{H}^\top \vec{y},
\end{equation}
where $y_j$ is the $j$-th element of $\vec{y}$. This is an expression for bending displacement interpolation functions for $v,w$, and by simply writing $v,w$ in place of $y$, we may express $v,w$ as follows.
\begin{eqnarray}
v(x) = \sum_{j=1}^4 H_j(x)v_j = \vec{H}^\top \vec{v}, \\
w(x) = \sum_{j=1}^4 H_j(x)w_j = \vec{H}^\top \vec{w}.
\end{eqnarray}

We wish to apply these interpolating approximations to the Euler-Lagrange equations found as Eqns.~\ref{eq:EL_s_final},~\ref{eq:EL_v_final},~\ref{eq:EL_w_final}. Unfortunately, we have a fourth order spatial derivative in both $v$ and $w$, but $v$ and $w$ are approximated by third degree polynomials. To remedy this, we are inclined to use integration by parts to reduce the order of the derivatives. Integration by parts done when products of functions are inside an integral. Following this logic, we shall multiply the equations by their respective displacement direction interpolation functions as follows.
\begin{eqnarray}
[\rho A(\ddot s - 2\Omega\dot v-\Omega^2s-\dot\Omega v)-EAs_{xx}]\cdot s = \rho A\Omega^2(a+x)\cdot s, \\
\left(\rho A(\ddot v + 2\Omega\dot s-\Omega^2v+\dot\Omega s)+EI_{zz}v_{xxxx}-\rho A\Omega^2\frac{\partial}{\partial x}\left\lbrace\left[a(L-x)\vphantom{\frac{0}{0}}\right.\right.\right.\\
\left.\left.\left.+\frac{1}{2}(L^2-x^2)\right]\frac{\partial v}{\partial x}\right\rbrace\right)\cdot v = [p_v - \rho A\dot\Omega(a+x)]\cdot v, \nonumber \\
\left(\rho A\ddot w+EI_{yy}w_{xxxx}-\rho A\Omega^2\frac{\partial}{\partial x}\left\lbrace\left[a(L-x)\vphantom{\frac{0}{0}}\right.\right.\right.\\
\left.\left.\left.+\frac{1}{2}(L^2-x^2)\right]\frac{\partial w}{\partial x}\right\rbrace\right)\cdot w = p_w\cdot w. \nonumber
\end{eqnarray}
We now integrate each equation over the element. First consider the stretch coordinate $s$.
\begin{equation}
\int_0^l \rho A(\ddot s-2\Omega\dot v-\Omega^2s-\dot\Omega v)s-EAs_{xx}s \text{ d}x = \int_0^l \rho A\Omega^2(a+x)s\text{ d}x.
\end{equation}
The interpolation function for $s$ is a first degree polynomial, and we have a second order derivative. We will employ integration by parts to alleviate this situation.
Consider
\begin{equation}
\int_0^l -EAs_{xx}s\text{ d}x. 
\end{equation}
Let
\begin{equation}
\alpha = -EAs,
\end{equation}
then
\begin{equation}
\text d\alpha = \frac{\text d}{\text dx}(-EAs)\text{ d}x.
\end{equation}
Let
\begin{equation}
\text{d}\beta = s_{xx}\text{ d}x,
\end{equation}
then
\begin{equation}
\beta = \int \frac{\text{d}}{\text{d}x}s_x\text{ d}x = s_x.
\end{equation}
Then, by the integration by parts formula,
\begin{equation}
\int_0^l -EAs_{xx}s\text{ d}x = \left.\vphantom{\int}-EAss_x\right|_0^l+\int_0^l EAs_x^2\text{ d}x.
\label{eq:s.disc.int.by.parts}
\end{equation}
The first term on the right vanishes on the boundaries, so we have
\begin{equation}
-\int_0^l EAs_{xx}s\text{ d}x = \int_0^l EAs_x^2\text{ d}x.
\end{equation}
Applying this result, we have
\begin{equation}
\int_0^l \rho A(\ddot s-2\Omega\dot v-\Omega^2s-\dot\Omega v)s+EAs_x^2 \text{ d}x = \int_0^l \rho A\Omega^2(a+x)s\text{ d}x.
\label{eq:s_displacement_integral_nonsense}
\end{equation}
Immediately an inconsistency emerges. We have mixed terms in $s,v$, but the vector forms of $s,v$ do not allow for the other terms. We now expand our definitions of $s,v$ into one combined form so that this equation has a sensible definition. At the same time, for ease of notation and clarity, we shall roll the $w$ term into the combined term. Define the combined displacement vector $\vec d$ as follows.
\begin{equation}
\vec d = 
\begin{pmatrix}
s_i \\ 
v_i \\ 
\theta_i \\
w_i \\
\phi_i \\
s_{i+1} \\
v_{i+1} \\
\theta_{i+1} \\
w_{i+1} \\
\phi_{i+1}
\end{pmatrix}.
\end{equation}
Next, we define a combined interpolation vector $\vec{N}$ as follows.
\begin{equation}
\vec{N} = 
\begin{pmatrix}
H_{s1}(x) \\
H_{v1}(x) \\
H_{v2}(x) \\
H_{w1}(x) \\
H_{w2}(x) \\
H_{s2}(x) \\
H_{v3}(x) \\
H_{v4}(x) \\
H_{w3}(x) \\
H_{w4}(x) \\
\end{pmatrix},
\end{equation}
where the subindices $s,v,w$ indicate the displacement direction and $1,2,3,4$ indicate the index of the interpolatory vector from each displacement direction formulation. The vector $\vec N$ can be expressed as the sum of the directional vectors $\vec{N}_s,\vec{N}_v,\vec{N}_w$ as follows.
\begin{equation}
\vec{N} = \vec{N}_s + \vec{N}_v + \vec{N}_w,
\end{equation}
where the directional vectors are defined as follows.
\begin{eqnarray}
\vec{N}_s =
\begin{pmatrix}
H_{s1}(x) & 0 & 0 & 0 & 0 & H_{s2}(x) & 0 & 0 & 0 & 0
\end{pmatrix}^\top \\
\vec{N}_v =
\begin{pmatrix}
0 & H_{v1}(x) & H_{v2}(x) & 0 & 0 & 0 & H_{v3}(x) & H_{v4}(x) & 0 & 0
\end{pmatrix}^\top \\
\vec{N}_w =
\begin{pmatrix}
0 & 0 & 0 & H_{w1}(x) & H_{w2}(x) & 0 & 0 & 0 & H_{w3}(x) & H_{w4}(x)
\end{pmatrix}^\top .
\end{eqnarray}
We can now write the functions $s,v$ terms of the combined displacement vector $\vec d$ and the respective interpolatory vectors $\vec N_s,\vec N_v$ as follows.
\begin{equation}
s = \vec N_s^\top \vec d,
\end{equation}
\begin{equation}
v = \vec N_v^\top \vec d.
\end{equation}
\begin{equation}
w = \vec N_w^\top \vec d.
\end{equation}
We may now use Eqn.~\ref{eq:s_displacement_integral_nonsense} sensibly. Next, we analyze the bending coordinate $v$. Before we can integrate over the element, we must recall the conditions of the integral 
\begin{equation}
\int_x^L(a+\eta)\text{ d}\eta.
\end{equation}
This integral treats $x$ on the full coordinate system, not just in the element. To consider the element, recall that
\begin{equation}
\int_a^bf(x)\text{ d}x = \int_0^{b-a}f(x+a)\text{ d}x.
\end{equation}
In our case,
\begin{equation}
f(x) = \int_x^L a+\eta\text{ d}\eta,
\end{equation}
so that under the revised coordinates, we transform the elemental integration from the limits $x_e\rightarrow x_{e+1}$ to $0\rightarrow l$ for $l = x_{e+1}-x_e$. For $f$, we have
\begin{equation}
f(x+x_e) = \int_{x+x_e}^L a+\eta \text{ d}\eta = a(L-(x+x_e))+\frac{1}{2}(L^2-(x+x_e)^2).
\end{equation}
It is not apparent, but the displacement approximation polynomials were already described in the space $0\rightarrow l$ instead of $x_e\rightarrow x_{e+1}$, so we don't need to adjust $s,v,w$. We only require a modification to the function $f$ as described above. Thus, the $v$ direction is expressed as
\begin{eqnarray}
\int_0^l\rho A(\ddot v + 2\Omega\dot s-\Omega^2v+\dot\Omega s)v+EI_{zz}v_{xxxx}v-\rho A\Omega^2\left[a(L-(x+x_e))\vphantom{\frac{0}{0}}\right.\\
\left.+\frac{1}{2}(L^2-(x+x_e)^2)\right]\frac{\partial^2 v}{\partial x^2} v\text{ d}x = \int_0^l[p_v - \rho A\dot\Omega(a+x)]v\text{ d}x. \nonumber
\end{eqnarray}
This equation has two terms to be simplified, the fourth order spatial derivative and the third term which must be simplified. Integration by parts twice on the fourth order term yields
\begin{equation}
\int_0^lEI_{zz}\frac{\partial^4 v}{\partial x^4}v\text{ d}x = \left.EI_{zz}v\frac{\partial^3 v}{\partial x^3}\right|_0^l-\left.EI_{zz}\frac{\partial v}{\partial x}\frac{\partial^2 v}{\partial x^2}\right|_0^l + \int_0^lEI_{zz}v_{xx}^2\text{ d}x,
\label{eq:v.disc.int.by.parts.1}
\end{equation}
and the first two terms on the right cancel due to boundary conditions, yielding
\begin{equation}
\int_0^lEI_{zz}\frac{\partial^4 v}{\partial x^4}v\text{ d}x =  \int_0^lEI_{zz}v_{xx}^2\text{ d}x.
\end{equation}
The third term is manipulated as follows. Consider
\begin{equation}
\int_0^l-\rho A\Omega^2\frac{\partial}{\partial x}\left\lbrace \left[a(L-(x+x_e))+\frac{1}{2}(L^2-(x+x_e)^2)\right]\frac{\partial v}{\partial x}\right\rbrace v\text{ d}x.
\end{equation}
Let
\begin{equation}
\alpha = -\rho A\Omega^2v,
\end{equation}
then
\begin{equation}
\text d\alpha = \frac{\text d}{\text dx}(-\rho A\Omega^2v)\text{ d}x = -\rho A\Omega^2\frac{\partial v}{\partial x}.
\end{equation}
Let
\begin{equation}
\text d\beta = \frac{\partial}{\partial x}\left\lbrace \left[a(L-(x+x_e))+\frac{1}{2}(L^2-(x+x_e)^2)\right]\frac{\partial v}{\partial x}\right\rbrace\text{ d}x,
\end{equation}
then
\begin{equation}
\beta = \left[a(L-(x+x_e))+\frac{1}{2}(L^2-(x+x_e)^2)\right]\frac{\partial v}{\partial x}.
\end{equation}
The integration by parts formula yields
\begin{eqnarray}
\label{eq:v.disc.int.by.parts.2}
& &\int_0^l-\rho A\Omega^2\frac{\partial}{\partial x}\left\lbrace \left[a(L-(x+x_e))+\frac{1}{2}(L^2-(x+x_e)^2)\right]\frac{\partial v}{\partial x}\right\rbrace v\text{ d}x\nonumber\\
&=&-\left.v\left[a(L-(x+x_e))+\frac{1}{2}(L^2-(x+x_e)^2)\right]\frac{\partial v}{\partial x}\right|_0^l\\
&+&\int_0^l \rho A\Omega^2\left[a(L-(x+x_e))+\frac{1}{2}(L^2-(x+x_e)^2)\right]v_x^2\text{ d}x, \nonumber
\end{eqnarray}
where the first term on the right side vanishes due to boundary conditions, yielding
\begin{eqnarray}
& &-\int_0^l\rho A\Omega^2\frac{\partial}{\partial x}\left\lbrace \left[a(L-(x+x_e))+\frac{1}{2}(L^2-(x+x_e)^2)\right]\frac{\partial v}{\partial x}\right\rbrace v\text{ d}x\\
&=& \int_0^l \rho A\Omega^2\left[a(L-(x+x_e))+\frac{1}{2}(L^2-(x+x_e)^2)\right]v_x^2\text{ d}x. \nonumber
\end{eqnarray}
The full $v$ bending integral equation is written as
\begin{eqnarray}
\label{eq:v_bending_integral_eqn}
& &\int_0^l\rho A(\ddot v + 2\Omega\dot s-\Omega^2v+\dot\Omega s)v+EI_{zz}v_{xx}^2\\
&+&\rho A\Omega^2\left[a(L-(x+x_e))+\frac{1}{2}(L^2-(x+x_e)^2)\right]v_x^2\text{ d}x = \int_0^l[p_v - \rho A\dot\Omega(a+x)]v\text{ d}x. \nonumber
\end{eqnarray}
The mixed terms are already handled by the combined displacement vector $\vec d$ as well as the interpolation vectors $\vec N_s,\vec N_v$

\begin{comment}
The full $v$ bending integral equation is written as
\begin{eqnarray}
\label{eq:v_bending_integral_eqn}
\int_0^l\rho A(\ddot v + 2\Omega\dot s-\Omega^2v+\dot\Omega s)v+EI_{zz}v_{xx}^2+\\
\rho A\Omega^2\left[a(L-(x+x_e))+\frac{1}{2}(L^2-(x+x_e)^2)\right]v_x^2\text{ d}x = \int_0^l[p_v - \rho A\dot\Omega(a+x)]v\text{ d}x. \nonumber
\end{eqnarray}
The mixed terms are already handled by the combined displacement vector $\vec d$ as well as the interpolation vectors $\vec N_s,\vec N_v$.
\end{comment}

The integral equation for the $w$ bending direction is handled similarly to the $v$ bending direction. There is a fourth order term and the third term that can be simplified. The process is exactly the same with $w$ written in place of $v$. The result is written as follows.
\begin{eqnarray}
& &\int_0^l\rho A\ddot ww+EI_{zz}w_{xx}^2+\rho A\Omega^2\left[a(L-(x+x_e))\vphantom{\frac{0}{0}}\right. \label{eq:w.bending.integral.eqn}\\
&+& \left.\frac{1}{2}(L^2-(x+x_e)^2)\right]w_x^2\text{ d}x = \int_0^lp_ww\text{ d}x.\nonumber 
\end{eqnarray}

From the application of the discretization to the equations of motion, we now have the essential boundary conditions. We shall show that the boundary conditions are met by the approximating functions $s(x), v(x), w(x)$.

From Eqn.~\ref{eq:s.disc.int.by.parts}, we have that 
\begin{equation}
\left.s\frac{\partial s}{\partial x}\right|_0^L = 0.
\end{equation}
From the physical system, we know that at $x=0$, the beam has zero displacement in each displacement direction. However, at the end of the beam ($x=L$), the beam may certainly be displaced in every direction. We conclude, then, that this condition is satisfied if $s_x(L) = 0$. 

The formulation of $v(x),w(x)$ is exactly the same, so the essential boundary conditions are also the same. For either case, we define $\xi(x), \xi\in\{v,w\}$ for ease of discussion. We have from Eqn.~\ref{eq:v.disc.int.by.parts.1}
\begin{equation}
\left.\xi\frac{\partial^3 \xi}{\partial x^3}\right|_0^L = 0
\end{equation}
and  
\begin{equation}
\left.\frac{\partial \xi}{\partial x}\frac{\partial^2 \xi}{\partial x^2}\right|_0^L = 0,
\end{equation}
and from Eqn.~\ref{eq:v.disc.int.by.parts.2}
\begin{equation}
\left.\xi\left[a(L-x)+\frac{1}{2}(L^2-x^2)\right]\frac{\partial \xi}{\partial x}\right|_0^L = 0.
\end{equation}
As in the case of $s$, the displacement at $x=0$ must be zero; additionally, the bending must be zero. Thus we have $\xi(0)=0,\xi_x(0)=0$. For the first condition, we must have that $\xi_{xxx}(L) = 0$, which represents the vanishing shear force for a free beam; indeed this agrees with the physical system. For the second condition, we must have that $\xi_{xx}(L) = 0$, which represents the vanishing of the bending moment for a free beam; this also agrees with our physical system. The third condition is zero at $x=L$, so there is no more work to do. 

The partial derivatives applied to the vector form of $s,v,w$ can be expressed in the following manner.
\begin{eqnarray}
z_x = [\vec N_z]_x^\top\vec d, \\
z_{xx} = [\vec N_z]_{xx}^\top\vec d, \\
\dot z = [\vec N_z]^\top\dot{\vec d}, \\
\ddot z = [\vec N_z]^\top\ddot{\vec d},
\end{eqnarray}
where $z$ is $s,v$ or $w$ and the subscripts $x,xx$ indicate derivative with respect to $x$ and the second derivative with respect to $x$, respectively. The over dot represents differentiation with respect to time. The differentiation operation on the vector applies the derivative to each element of the vector and results in a vector of the same size. 

Under this new notation, the $s$ direction integral equation (Eqn.~\ref{eq:s_displacement_integral_nonsense}) is written as
\begin{eqnarray}
\int_0^l \rho A\left(\vec{N_s}^\top\ddot{\vec d}-2\Omega\vec{N_v}^\top\dot{\vec{d}}-\Omega^2\vec{N_s}^\top\vec{d}-\dot\Omega \vec{N_v}^\top\vec{d}\right)\vec{N_s}^\top\vec{d}+EA\left([\vec{N_s}]_x^\top\vec{d}\right)^2 \text{ d}x  \\
= \int_0^l \rho A\Omega^2(a+x)\vec{N_s}^\top\vec{d}\text{ d}x. \nonumber
\end{eqnarray}
Observe that
\begin{equation}
\left([\vec{N_s}]_x^\top\vec{d}\right)^2 = 
\left([\vec{N_s}]_x^\top\vec{d}\right)\left([\vec{N_s}]_x^\top\vec{d}\right) = 
\left([\vec{N_s}]_x^\top\vec{d}\right)[\vec{N_s}]_x^\top\vec{d} = 
\left([\vec{N_s}]_x^\top\vec{d}[\vec{N_s}]_x^\top\right)\vec{d}.
\label{eq:vec_prod_square}
\end{equation}
With this result, we may pull $\vec d$ out to the right on all terms, which yields
\begin{eqnarray}
\int_0^l \left\lbrace\rho A\left(\vec{N_s}^\top\ddot{\vec d}-2\Omega\vec{N_v}^\top\dot{\vec{d}}-\Omega^2\vec{N_s}^\top\vec{d}-\dot\Omega \vec{N_v}^\top\vec{d}\right)\vec{N_s}^\top\right. \\
\left.+EA\left([\vec{N_s}]_x^\top\vec{d}[\vec{N_s}]_x^\top\right) \right\rbrace\vec{d}\text{ d}x = \int_0^l \left\lbrace\rho A\Omega^2(a+x)\vec{N_s}^\top\right\rbrace\vec{d}\text{ d}x. \nonumber
\end{eqnarray}
\begin{lemma}
Given nontrivial vectors $\{\alpha,\beta,\gamma\}\in\mathbb{R}^n,$
\begin{equation}
\int\vec{\alpha}^\top\vec{\gamma}\text{ d}\xi = \int\vec{\beta}^\top\vec{\gamma}\text{ d}\xi \Leftrightarrow \int\vec{\alpha}^\top\text{ d}\xi = \int\vec{\beta}^\top\text{ d}\xi.
\end{equation}
\end{lemma}
\begin{proof}
\begin{equation}
\begin{array}{rrclr}
&\int\vec{\alpha}^\top\vec{\gamma}\text{ d}\xi &=& \int\vec{\beta}^\top\vec{\gamma}\text{ d}\xi &\\
\Leftrightarrow& \vec{\alpha}^\top\vec{\gamma} &=& \vec{\beta}^\top\vec{\gamma} &\\
\Leftrightarrow& \left(\vec{\alpha}^\top -  \vec{\beta}^\top\right)\vec{\gamma} &=& 0 &\\
\Leftrightarrow& \left(\vec{\alpha}^\top-\vec{\beta}^\top\right) &=& 0 &(\vec{\gamma}\text{ is nontrivial}) \\
\Leftrightarrow& \vec{\alpha}^\top &=& \vec{\beta}^\top &\\
\Leftrightarrow& \int\vec{\alpha}^\top\text{ d}\xi &=& \int\vec{\beta}^\top\text{ d}\xi &
\end{array}
\end{equation}
\end{proof}
The integral of the vector is not defined. Let this notation imply that we shall integrate each element of the vector. Since $\vec d$ is non-trivial, we have
\begin{eqnarray}
\label{eq:stretch_integral_eqn_disorg}
\int_0^l \rho A\left(\vec{N_s}^\top\ddot{\vec d}-2\Omega\vec{N_v}^\top\dot{\vec{d}}-\Omega^2\vec{N_s}^\top\vec{d}-\dot\Omega \vec{N_v}^\top\vec{d}\right)\vec{N_s}^\top+EA[\vec{N_s}]_x^\top\vec{d}[\vec{N_s}]_x^\top\text{ d}x \\
= \int_0^l \rho A\Omega^2(a+x)\vec{N_s}^\top\text{ d}x. \nonumber
\end{eqnarray}
Consider the vectors $\left\lbrace\vec{\alpha},\vec{\beta},\vec{\gamma}\right\rbrace\in\mathbb{R}^n$ of the same length. Note that
\begin{equation}
\left(\vec\alpha^\top\vec\gamma\right)\vec\beta^\top = \left[\vec\beta\left(\vec\gamma^\top\alpha\right)\right]^\top =  \left[\left(\vec\beta\vec\alpha^\top\right)\vec\gamma\right]^\top.
\label{eq:vec_prod_transpose}
\end{equation}
We shall use this property to organize Eqn.~\ref{eq:stretch_integral_eqn_disorg}. At the same time, we shall group the terms by time derivatives of the vector $\vec d$. We have
\begin{eqnarray}
\label{eq:s.disc.final}
\int_0^l \left\lbrace\rho A\vec{N_s}\vec{N_s}^\top\ddot{\vec d}-\rho A2\Omega\vec{N_s}\vec{N_v}^\top\dot{\vec{d}}\right. \nonumber \\
\left.+\left(\rho A\left[-\Omega^2\vec{N_s}\vec{N_s}^\top-\dot\Omega \vec{N_s}\vec{N_v}^\top\right]+EA[\vec{N_s}]_x[\vec{N_s}]_x^\top\right)\vec{d}\right\rbrace\text{ d}x \\
 = \int_0^l \rho A\Omega^2(a+x)\vec{N_s}\text{ d}x. \nonumber
\end{eqnarray}
The $v$ bending integral equation (Eqn.~\ref{eq:v_bending_integral_eqn}) may be discretized in a similar fashion.
\begin{eqnarray}
\int_0^l\left\lbrace\rho A\left(\vec{N_v}^\top\ddot{\vec d} + 2\Omega\vec{N_s}^\top\dot{\vec d}-\Omega^2\vec{N_v}^\top\vec{d}+\dot\Omega \vec{N_s}^\top\vec{d}\right)\vec{N_v}^\top\vec{d}\vphantom{\left([\vec{N_v}]_{x}^\top\vec{d}\right)^2}\right.\nonumber \\
\left.+EI_{zz}\left([\vec{N_v}]_{xx}^\top\vec{d}\right)^2+\rho A\Omega^2\left[a(L-x)+\frac{1}{2}(L^2-x^2)\right]\left([\vec{N_v}]_{x}^\top\vec{d}\right)^2\right\rbrace\text{ d}x \\
= \int_0^l[p_v - \rho A\dot\Omega(a+x)]\vec{N_v}^\top\vec{d}\text{ d}x. \nonumber
\end{eqnarray}
Using the observation of Eqn.~\ref{eq:vec_prod_square} and ``cancelling'' the non-trivial vector $\vec d$ yields
\begin{eqnarray}
\int_0^l\left\lbrace\rho A\left(\vec{N_v}^\top\ddot{\vec d} + 2\Omega\vec{N_s}^\top\dot{\vec d}-\Omega^2\vec{N_v}^\top\vec{d}+\dot\Omega \vec{N_s}^\top\vec{d}\right)\vec{N_v}^\top\vphantom{\frac{0}{0}}\right.\nonumber \\
\left.+EI_{zz}[\vec{N_v}]_{xx}^\top\vec{d}[\vec{N_v}]_{xx}^\top+\rho A\Omega^2\left[a(L-x)+\frac{1}{2}(L^2-x^2)\right][\vec{N_v}]_{x}^\top\vec{d}[\vec{N_v}]_{x}^\top\right\rbrace\text{ d}x \\
= \int_0^l[p_v - \rho A\dot\Omega(a+x)]\vec{N_v}^\top\text{ d}x. \nonumber
\end{eqnarray}
Applying the observation of Eqn.~\ref{eq:vec_prod_transpose} to reorganize, we have
\begin{eqnarray}
\label{eq:v.disc.final}
\int_0^l\left\lbrace\rho A\vec{N_v}\vec{N_v}^\top\ddot{\vec d} + \rho A 2\Omega\vec{N_v}\vec{N_s}^\top\dot{\vec d}+\left(\rho A[-\Omega^2\vec{N_v}\vec{N_v}^\top+\dot\Omega \vec{N_v}\vec{N_s}^\top]\vphantom{\frac{0}{0}}\right.\right.\nonumber \\
\left.\left.+EI_{zz}[\vec{N_v}]_{xx}[\vec{N_v}]_{xx}^\top+\rho A\Omega^2\left[a(L-x)+\frac{1}{2}(L^2-x^2)\right][\vec{N_v}]_{x}[\vec{N_v}]_{x}^\top\right)\vec{d}\right\rbrace\text{ d}x \\
= \int_0^l[p_v - \rho A\dot\Omega(a+x)]\vec{N_v}\text{ d}x. \nonumber
\end{eqnarray}
Similarly, the $w$ integral equation (Eqn.~\ref{eq:w.bending.integral.eqn}) may be discretized.
\begin{eqnarray}
\int_0^l\left\lbrace\rho A\vec{N_w}^\top\ddot{\vec{d}}\vec{N_w}^\top\vec{d}+EI_{yy}\left([\vec{N_w}]_{xx}^\top\vec{d}\right)^2\vphantom{\frac{0}{0}}\right.  \\
\left.+\rho A\Omega^2\left[a(L-x)+\frac{1}{2}(L^2-x^2)\right]\left([\vec{N_w}]_{x}^\top\vec{d}\right)^2\right\rbrace\text{ d}x 
= \int_0^lp_w\vec{N_w}^\top\vec{d}\text{ d}x. \nonumber 
\end{eqnarray}
Using the observation of Eqn.~\ref{eq:vec_prod_square} and ``cancelling'' the non-trivial vector $\vec d$ yields
\begin{eqnarray}
\int_0^l\left\lbrace\rho A\vec{N_w}^\top\ddot{\vec{d}}\vec{N_w}^\top+EI_{yy}[\vec{N_w}]_{xx}^\top\vec{d}[\vec{N_w}]_{xx}^\top\vphantom{\frac{0}{0}}\right.  \\
\left.+\rho A\Omega^2\left[a(L-x)+\frac{1}{2}(L^2-x^2)\right][\vec{N_w}]_{x}^\top\vec{d}[\vec{N_w}]_{x}^\top\right\rbrace\text{ d}x 
= \int_0^lp_w\vec{N_w}^\top\text{ d}x. \nonumber 
\end{eqnarray}
Applying the observation of Eqn.~\ref{eq:vec_prod_transpose} to reorganize, we have
\begin{eqnarray}
\label{eq:w.disc.final}
\int_0^l\left\lbrace\rho A\vec{N_w}\vec{N_w}^\top\ddot{\vec{d}}+\left(EI_{yy}[\vec{N_w}]_{xx}[\vec{N_w}]_{xx}^\top\vphantom{\frac{0}{0}}\right.\right.  \\
\left.\left.+\rho A\Omega^2\left[a(L-x)+\frac{1}{2}(L^2-x^2)\right][\vec{N_w}]_{x}[\vec{N_w}]_{x}^\top\right)\vec{d}\right\rbrace\text{ d}x 
= \int_0^lp_w\vec{N_w}\text{ d}x. \nonumber 
\end{eqnarray}
We now combine Eqn.~\ref{eq:s.disc.final}, Eqn.~\ref{eq:v.disc.final} and Eqn.~\ref{eq:w.disc.final} to form the full system, grouping the terms by time derivatives of $\vec{d}$.
\begin{eqnarray}
\label{eq:eom.unwieldy}
\int_0^l \left\lbrace\rho A\left[\vec{N_s}\vec{N_s}^\top+\vec{N_v}\vec{N_v}^\top+\vec{N_w}\vec{N_w}^\top\right]\ddot{\vec d}+\rho A2\Omega\left[\vec{N_v}\vec{N_s}^\top-\vec{N_s}\vec{N_v}^\top\right]\dot{\vec{d}}\right. \nonumber \\
\left.+\left(-\rho A\Omega^2\left[\vec{N_s}\vec{N_s}^\top+\vec{N_v}\vec{N_v}^\top\right]+\rho A\dot\Omega \left[\vec{N_v}\vec{N_s}^\top-\vec{N_s}\vec{N_v}^\top\right] +EA[\vec{N_s}]_x[\vec{N_s}]_x^\top\right.\right.\nonumber \\
\left.\left.+\rho A\Omega^2\left[a(L-x)+\frac{1}{2}(L^2-x^2)\right]\left[\vphantom{\frac{0}{0}}[\vec{N_v}]_{x}[\vec{N_v}]_{x}^\top+[\vec{N_w}]_{x}[\vec{N_w}]_{x}^\top\right]\right.\right.\\
\left.\left. +E\left[I_{zz}[\vec{N_v}]_{xx}[\vec{N_v}]_{xx}^\top+I_{yy}[\vec{N_w}]_{xx}[\vec{N_w}]_{xx}^\top\right]\right)\vec{d}\right\rbrace\text{ d}x \nonumber \\
 = \int_0^l \rho A\Omega^2(a+x)\vec{N_s}+[p_v - \rho A\dot\Omega(a+x)]\vec{N_v}+p_w\vec{N_w}\text{ d}x. \nonumber 
\end{eqnarray}
We have the equations of motion for the discretized system represented by Eqn.~\ref{eq:eom.unwieldy}, although it may not be very clear. To clarify this expression, first note that 
\begin{equation}
\int_0^l\vec d\text{ d}x = \int_0^l\text{ d}x\vec d,
\end{equation}
and the same applies to the time derivatives $\ddot{\vec d},\dot{\vec d}$, since the elements of these vectors are constant with respect to $x$ (they are nodal DOF). We now define element matrices to simplify the expression and illuminate the equations of motion. Let
\begin{eqnarray}
\mathbf{[p]} &=& \rho A\int_0^l\vec{N_s}\vec{N_s}^\top+\vec{N_v}\vec{N_v}^\top\text{ d}x \\
\mathbf{[m]} &=& \mathbf{[p]}+\rho A\int_0^l\vec{N_w}\vec{N_w}^\top\text{ d}x, \\
\mathbf{[g]} &=& \rho A\int_0^l\vec{N_v}\vec{N_s}^\top-\vec{N_s}\vec{N_v}^\top\text{ d}x ,\\
\mathbf{[k]} &=& E\int_0^lA[\vec{N_s}]_x[\vec{N_s}]_x^\top+I_{zz}[\vec{N_v}]_{xx}[\vec{N_v}]_{xx}^\top+I_{yy}[\vec{N_w}]_{xx}[\vec{N_w}]_{xx}^\top\text{ d}x, 
\end{eqnarray}
\begin{eqnarray}
\mathbf{[\sigma]} &=& \rho A\int_0^l \Omega^2\left[a(L-x)+\frac{1}{2}(L^2-x^2)\right]\left[\vphantom{\frac{0}{0}}[\vec{N_v}]_{x}[\vec{N_v}]_{x}^\top+[\vec{N_w}]_{x}[\vec{N_w}]_{x}^\top\right] \text{ d}x, \\
\vec{f} &=& \int_0^l \rho A\Omega^2(a+x)\vec{N_s}+[p_v - \rho A\dot\Omega(a+x)]\vec{N_v}+p_w\vec{N_w}\text{ d}x.
\end{eqnarray}
Then we may express Eqn.~\ref{eq:eom.unwieldy} as
\begin{equation}
\mathbf{[m]}\ddot{\vec d} +2\Omega\mathbf{[g]}\dot{\vec d} + (\mathbf{[k]}+\Omega^2(\mathbf{[\sigma]}-\mathbf{[p]})+\dot\Omega\mathbf{[g]} )\vec d = \vec{f}.
\label{eq:eom.elt.mx}
\end{equation}
The matrices in Eqn.~\ref{eq:eom.elt.mx} are related to the physics and are given special names. The matrix $\mathbf{[m]}$ is called the \textbf{mass matrix}, $\mathbf{[k]}$ is called the \textbf{stiffness matrix}, and $\mathbf{[g]}$ is called the \textbf{gyroscopic matrix}. Note that the total stiffness depends on $\mathbf{[\sigma]},\mathbf{[p]},$ and $\mathbf{[g]}$, where the term involving $\mathbf{[\sigma]}$ and $\mathbf{[p]}$ describes Coriolis stiffening, while the term involving $\mathbf{[g]}$ describes inertial stiffening. If the cantilever beam is not rotating (\emph{i.e.} $\Omega = 0$) we recover the standard cantilever beam system
\begin{equation}
\mathbf{[m]}\ddot{\vec d}+\mathbf{[k]}\vec d = \int_0^l p_v\vec{N_v}+p_w\vec{N_w}\text{ d}x.
\end{equation}

The computed matrices are given below.
\[
\mathbf{[p]} = 
\frac{l\rho A}{420}\begin{pmatrix}
 140 &0 & 0 & 0 & 0 &  70 &0 & 0 & 0 & 0\\
   0 &   156 &   22l & 0 & 0 &   0 &    54 &  -13l & 0 & 0\\
   0 &  22l &  4l^2 & 0 & 0 &   0 &  13l & -3l^2 & 0 & 0\\
   0 &0 & 0 & 0 & 0 &   0 &0 & 0 & 0 & 0\\
   0 &0 & 0 & 0 & 0 &   0 &0 & 0 & 0 & 0\\
  70 &0 & 0 & 0 & 0 & 140 &0 & 0 & 0 & 0\\
   0 &    54 &   13l & 0 & 0 &   0 &   156 &  -22l & 0 & 0\\
   0 & -13l & -3l^2 & 0 & 0 &   0 & -22l &  4l^2 & 0 & 0\\
   0 &0 & 0 & 0 & 0 &   0 &0 & 0 & 0 & 0\\
   0 &0 & 0 & 0 & 0 &   0 &0 & 0 & 0 & 0\\
\end{pmatrix}
\]
\[
\mathbf{[m]} = 
\frac{l\rho A}{420}\begin{pmatrix}
 140 &0 & 0 &0 & 0 &  70 &0 & 0 &0 & 0\\
   0 &   156 &   22l &0 & 0 &   0 &    54 &  -13l &0 & 0\\
   0 &  22l &  4l^2 &0 & 0 &   0 &  13l & -3l^2 &0 & 0\\
   0 &0 & 0 &   156 &   22l &   0 &0 & 0 &    54 &  -13l\\
   0 &0 & 0 &  22l &  4l^2 &   0 &0 & 0 &  13l & -3l^2\\
  70 &0 & 0 &0 & 0 & 140 &0 & 0 &0 & 0\\
   0 &    54 &   13l &0 & 0 &   0 &   156 &  -22l &0 & 0\\
   0 & -13l & -3l^2 &0 & 0 &   0 & -22l &  4l^2 &0 & 0\\
   0 &0 & 0 &    54 &   13l &   0 &0 & 0 &   156 &  -22l\\
   0 &0 & 0 & -13l & -3l^2 &   0 &0 & 0 & -22l &  4l^2\\
\end{pmatrix}
\]
\[
\mathbf{[g]} = 
\frac{l\rho A}{60} \begin{pmatrix}
    0 & -21 & -3l & 0 & 0 &    0 &  -9 & 2l & 0 & 0\\
   21 &   0 &    0 & 0 & 0 &    9 &   0 &   0 & 0 & 0\\
  3l &   0 &    0 & 0 & 0 &  2l &   0 &   0 & 0 & 0\\
    0 &   0 &    0 & 0 & 0 &    0 &   0 &   0 & 0 & 0\\
    0 &   0 &    0 & 0 & 0 &    0 &   0 &   0 & 0 & 0\\
    0 &  -9 & -2l & 0 & 0 &    0 & -21 & 3l & 0 & 0\\
    9 &   0 &    0 & 0 & 0 &   21 &   0 &   0 & 0 & 0\\
 -2l &   0 &    0 & 0 & 0 & -3l &   0 &   0 & 0 & 0\\
    0 &   0 &    0 & 0 & 0 &    0 &   0 &   0 & 0 & 0\\
    0 &   0 &    0 & 0 & 0 &    0 &   0 &   0 & 0 & 0\\
\end{pmatrix}
\]


\[
\mathbf{[k]} = 
\frac{E}{l^3}\begin{pmatrix}
  Al^2 &  0 &    0 &  0 &    0 & -Al^2 &   0 &    0 &   0 &    0\\
 0 &  12I_{zz} &   6I_{zz}l &  0 &    0 & 0 &  -12I_{zz} &   6I_{zz}l &   0 &    0\\
 0 & 6I_{zz}l & 4I_{zz}l^2 &  0 &    0 & 0 & -6I_{zz}l & 2I_{zz}l^2 &   0 &    0\\
 0 &  0 &    0 &  12I_{yy} &   6I_{yy}l & 0 &   0 &    0 &  -12I_{yy} &   6I_{yy}l\\
 0 &  0 &    0 & 6I_{yy}l & 4I_{yy}l^2 & 0 &   0 &    0 & -6I_{yy}l & 2I_{yy}l^2\\
 -Al^2 &  0 &    0 &  0 &    0 &  Al^2 &   0 &    0 &   0 &    0\\
 0 & -12I_{zz} &  -6I_{zz}l &  0 &    0 & 0 &   12I_{zz} &  -6I_{zz}l &   0 &    0\\
 0 & 6I_{zz}l & 2I_{zz}l^2 &  0 &    0 & 0 & -6I_{zz}l & 4I_{zz}l^2 &   0 &    0\\
 0 &  0 &    0 & -12I_{yy} &  -6I_{yy}l & 0 &   0 &    0 &   12I_{yy} &  -6I_{yy}l\\
 0 &  0 &    0 & 6I_{yy}l & 2I_{yy}l^2 & 0 &   0 &    0 & -6I_{yy}l & 4I_{yy}l^2\\
\end{pmatrix}
\]
Let 
\begin{eqnarray*}
\alpha_1 &=& L^2-x_e^2,\\
\alpha_2 &=& 2a(L-x_e)\\
\alpha_3 &=& l(a+x_e),\\
\beta_1 &=& -72l^2+252(\alpha_1+\alpha_2-\alpha_3),\\
\beta_2 &=& -15l^3+21l(\alpha_1+\alpha_2-2\alpha_3),\\
\beta_3 &=& 6l^3+21l(\alpha_1+\alpha_2),\\
\beta_4 &=& -4l^4+14l^2(2\alpha_1+2\alpha_2-\alpha_3),\\
\beta_5 &=& 3l^4+7l^2(-\alpha_1-\alpha_2+\alpha_3),\\
\beta_6 &=& -18l^4+14l^2(2\alpha_1+2\alpha_2-3\alpha_3),
\end{eqnarray*}
where $L$ is (global) beam length, $a$ is distance from the center of the axis of rotation to the beginning of the beam, $x_e$ is the position of the $e$-th node, and $l$ is the length of the element.
The matrix $\mathbf{[\sigma]}$ can be expressed as
\[
\mathbf{[\sigma]} = 
\frac{\rho A}{420l}\begin{pmatrix}
 0 & 0 &  0 & 0 &  0 & 0 & 0 & 0 & 0 &  0\\
 0 & \beta_1 & \beta_2 & 0 &  0 & 0 &  -\beta_1  & \beta_3 & 0 & 0\\
 0 & \beta_2 & \beta_4 & 0 &  0 & 0 & -\beta_2 & \beta_5 & 0 & 0\\
 0 & 0 & 0 & \beta_1 & \beta_2 & 0 & 0 & 0 &  -\beta_1  & \beta_3 \\
 0 & 0 & 0 & \beta_2 & \beta_4 & 0 & 0 & 0 & -\beta_2 & \beta_5 \\
 0 & 0 & 0 & 0 & 0 & 0 & 0 & 0 & 0 & 0\\
 0 & -\beta_1 & -\beta_2 & 0 & 0 & 0 & \beta_1 & -\beta_3 & 0 & 0\\
 0 & \beta_3 & \beta_5 & 0 & 0 & 0 & -\beta_3 & \beta_6 & 0 & 0\\
 0 & 0 & 0 & -\beta_1 & -\beta_2 & 0 & 0 & 0 & \beta_1 & -\beta_3 \\
 0 & 0 & 0 & \beta_3 & \beta_5  & 0 & 0 & 0 & -\beta_3 &  \beta_6
\end{pmatrix}.
\]
Naturally, all of the matrices are symmetric, except the matrix $\mathbf{[g]}$, which is skew-symmetric. In other words, 
\begin{equation}
\mathbf{[g]} = -\mathbf{[g]}^\top.
\end{equation}
\begin{lemma}
Skew symmetric matrices have purely imaginary eigenvalues (\emph{i.e.} $\text{Re}(\lambda)=0$). 
\end{lemma}
\begin{proof}
Let $\mathbf{A}$ be a real skew symmetric matrix, with eigenvalue $\lambda$ and eigenvector $\vec\xi$. Then, by definition,
\begin{equation}
\mathbf{A}\vec\xi = \lambda\vec\xi,
\label{eq:proof.eig}
\end{equation}
and 
\begin{equation}
\mathbf{A} = -\mathbf{A}^\top.
\end{equation}
Let $()^*$ denote the conjugate operator. We have
\begin{equation}
\begin{array}{rrclr}
 & (\mathbf{A}\vec\xi)^* &=& (\lambda\vec\xi)^* & \\
\Leftrightarrow & \mathbf{A}\vec\xi^* &=& \lambda^*\vec\xi^* & (\mathbf{A}\text{ is real}) \\
\Leftrightarrow & (\mathbf{A}\vec\xi^*)^\top &=& (\lambda^*\vec\xi^*)^\top & \\
\Leftrightarrow & (\mathbf{A}\vec\xi^*)^\top\vec\xi &=& (\lambda^*\vec\xi^*)^\top\vec\xi & \\
\Leftrightarrow & \vec\xi^{*\top}\mathbf{A}^\top\vec\xi &=& \lambda^*\vec\xi^{*\top}\vec\xi & \\
\Leftrightarrow & \vec\xi^{*\top}(\mathbf{-A})\vec\xi &=& \lambda^*\vec\xi^{*\top}\vec\xi & (\mathbf{A}\text{ is skew-symmetric}) \\
\Leftrightarrow & \vec\xi^{*\top}(-\lambda)\vec\xi &=& \lambda^*\vec\xi^{*\top}\vec\xi & (\text{Eqn.~\ref{eq:proof.eig}}) \\
\Leftrightarrow & -\lambda\vec\xi^{*\top}\vec\xi &=& \lambda^*\vec\xi^{*\top}\vec\xi & (\text{scalars commute}) \\
\Leftrightarrow & -\lambda\vec\xi^{*\top} &=& \lambda^*\vec\xi^{*\top} & (\vec\xi\text{ is nontrivial})\\
\Leftrightarrow & -\lambda &=& \lambda^* & (\vec\xi\text{ is nontrivial})\\
\Leftrightarrow & \text{Re}(\lambda) &=& 0 & \\
\end{array}
\end{equation}
\end{proof}

As a result of the configuration of $\mathbf{[g]}$, the contribution of the $\dot{\vec d}$ term does not add nor remove energy from the system --- and important distinction, as that would imply that the physical system would always decay to zero or grow unbounded, and we know the physical system does not do that.

Next, we must assemble the system. Consider Eqn.~\ref{eq:eom.elt.mx}. The matrices $\mathbf{m},\mathbf{g},\mathbf{k},\mathbf{\sigma},\mathbf{p}$ are in terms of the element. Each element is, essentially, the connection of two nodes. Each node has the five degrees of freedom: $s$ stretch, $v$ displacement, $v$ bending, $w$ displacement and $w$ bending. If we label the elements and nodes in an orderly fashion from the base of the beam to the tip, we find that element one is the connection of nodes one and two, element two is the connection of nodes two and three, and so on up to the last or $n$-th element, which is the connection of the $n$-th and $(n+1)$-th nodes. If the elements are all equal length, then the properties of each element are identical and we may create the element matrices once for our assembly. If not, the assembly loop will require us to build the element matrices at each iteration. In either case, we shall have a block diagonal matrix, which is densly populated near the diagonal, and zeros (or at least very sparse) elsewhere. This could be useful if the system is to be modeled with thousands of elements; routines are available in may environments (\emph{e.g.} MatLab, Python/NumPy, \emph{etc.}) which allow for optimized computation of sparse matrices so that much less memory is used.

Whatever the method of labeling and indexing, assembly is most generally accomplished by creating an index of degrees of freedom in the global frame. The $i$-th element will have ten degrees of freedom. The local degrees of freedom, indexed as $i=1,2,\dots,10$, correspond to the matrix entries in $\mathbf{m},\mathbf{g},\mathbf{k},\mathbf{\sigma},\mathbf{p}$. The indexing vector $\vec g_{\text{dofs}}$ stores the global degrees of freedom. These degrees of freedom, in turn, correspond to the matrix indices of $\mathbf{M},\mathbf{G},\mathbf{K},\mathbf{S},\mathbf{P}$, the global mass, gyroscopic, stiffness and Coriolis stiffening matrices. Construction of the indexing vector is done as
\begin{equation}
\vec g_{\text{dofs}}^i(j) = 5(i-1)+j,
\end{equation}
for the $i$-th element, where the $j$-th local index is in $\lbrace1,2,\dots,10\rbrace$. The $(j,k)$-th local matrix entry for the $i$-th element modifies the global matrix as follows.
\begin{equation}
\mathbf{K}(\vec{g}_{\text{dof}}^i(j),\vec{g}_{\text{dof}}^i(k)) = \mathbf{K}(\vec{g}_{\text{dof}}^i(j),\vec{g}_{\text{dof}}^i(k))+\mathbf{k}(j,k),
\end{equation}
for $j,k$ $\in\lbrace1,2,\dots,10\rbrace$. The process is the same for $\textbf{g}\rightarrow\textbf{G},\mathbf{p}\rightarrow\mathbf{P},\mathbf{m}\rightarrow\mathbf{M},\mathbf{\sigma}\rightarrow\mathbf{S}$. We then formulate the global system.
\begin{equation}
\mathbf{[M]}\ddot{\vec d} +2\Omega\mathbf{[G]}\dot{\vec d} + (\mathbf{[K]}+\Omega^2(\mathbf{[S]}-\mathbf{[P]})+\dot\Omega\mathbf{[G]} )\vec d = \vec{\mathbf{F}}
\end{equation}

\chapter{RESULTS}
\label{ch:results}
The test of the system is carried out by examining the natural modes of the discretization and comparing against published material. To do so, it is important to consider the system at equilibrium; in this case, we take angular acceleration to be zero so that $\dot\Omega=0$. We relabel the matrices as follows, consistent with the order of the derivative.
\begin{eqnarray}
\mathbf{X0} &=& \mathbf{[K]}+\Omega^2(\mathbf{[S]}-\mathbf{[P]}),\\
\mathbf{X1} &=& 2\Omega\mathbf{[G]},\\
\mathbf{X2} &=& \mathbf{[M]},
\end{eqnarray}
so that at equilibrium we have
\begin{equation}
\mathbf{[X2]}\ddot{\vec d}+\mathbf{[X1]}\dot{\vec d}+\mathbf{[X0]}\vec d=0.
\label{eq:equilib_system}
\end{equation}
To find the natural frequencies, we assume the ansatz
\begin{equation}
\vec d = \vec\xi e^{i\omega t},
\end{equation}
and plugging into Eqn.~\ref{eq:equilib_system} yields 
\begin{equation}
-\mathbf{[X2]}\omega^2\vec\xi e^{i\omega t}+\mathbf{[X1]}i\omega\vec\xi e^{i\omega t}+\mathbf{[X0]}\vec\xi e^{i\omega t}=0.
\end{equation}
Cancelling $e^{i\omega t}$ and factoring yields 
\begin{equation}
\left(-\mathbf{[X2]}\omega^2+\mathbf{[X1]}i\omega+\mathbf{[X0]}\right)\vec\xi=0.
\end{equation}
This equation is known as a \emph{quadratic} eigenvalue problem. Whereas the generalized eigenvalue problem has the form
\begin{equation}
\left(-\mathbf{[X2]}\omega^2+\mathbf{[X0]}\right)\vec\xi=0,
\end{equation}
written more typically as
\begin{equation}
\mathbf{[M]}\lambda=\mathbf{[K]},
\end{equation}
and highly optimized algorithms exist for this formulation. The algorithms for the quadratic eigenvalue problem recast themselves into the generalized eigenvalue problem using various methods. According to Ref.~\cite{Tisseur01thequadratic} the ``standard'' method for factorizing the system into its linear components, known as the \emph{first companion form}, is accomplished as follows. Let
\begin{equation}
u=\omega\xi,
\end{equation}
so that
\begin{equation}
-\mathbf{[X2]}u\omega+\mathbf{[X1]}iu+\mathbf{[X0]}\xi=0.
\end{equation}
The generalized eigenvalue problem may now be written as
\begin{equation}
\begin{pmatrix}
\mathbf{0} & \mathbf{I}\\
\mathbf{X0} & i\mathbf{X1}
\end{pmatrix}
\begin{pmatrix}
\xi \\
u
\end{pmatrix}
-\omega
\begin{pmatrix}
\mathbf{I} & \mathbf{0}\\
\mathbf{0} & \mathbf{X2}
\end{pmatrix}
\begin{pmatrix}
\xi \\
u
\end{pmatrix}
=0,
\end{equation}
where $\mathbf{0}$ is a zero filled matrix, and $\mathbf{I}$ is the identity matrix. In general, the first companion form is described by
\begin{equation}
\mathbf{L1:}
\begin{pmatrix}
\mathbf{0} & \mathbf{N}\\
\mathbf{-X0} & \mathbf{-X1}
\end{pmatrix}-\omega
\begin{pmatrix}
\mathbf{N} & \mathbf{0} \\
\mathbf{0} & \mathbf{X2},
\end{pmatrix}
\end{equation}
where $\mathbf{N}$ is any nonsingular matrix. Thus, it is often convenient to choose $\mathbf{N}=\mathbf{I}$, or some multiple of the identity. As one might expect, there are several companion forms. Of particular interest are the third and fourth companion forms, which are well suited for gyroscopic problems. These linearizations are described below.
\begin{equation}
\mathbf{L3:}
\begin{pmatrix}
\mathbf{X0} & \mathbf{0}\\
\mathbf{X1} & \mathbf{X0}
\end{pmatrix} -\omega
\begin{pmatrix}
\mathbf{0} & \mathbf{X0} \\
\mathbf{-X2} & \mathbf{0}
\end{pmatrix},
\end{equation}
\begin{equation}
\mathbf{L4:}
\begin{pmatrix}
\mathbf{0} & \mathbf{-X0}\\
\mathbf{X2} & \mathbf{0}
\end{pmatrix} -\omega
\begin{pmatrix}
\mathbf{X2} & \mathbf{X1} \\
\mathbf{0} & \mathbf{X2}
\end{pmatrix}.
\end{equation}
The third companion form is preferred when $\mathbf{X2}$ is singular or near singular, and the fourth companion form is preferred when $\mathbf{X0}$ is singular or near singular.

The three linearization techniques described above were tested for accuracy and the most accurate method was used throughout the testing. The accuracy was tested by considering eigenvalue spectrum of the result. The eigenvalues returned should either be purely imaginary, or real. If the eigenvalues are not either purely imaginary or real, the solution to the eigenvalue problem yields inaccurate results. Thus, the quality of the result can be calculated. The method employed in this calculation is given below.

\begin{lemma}
Given $\alpha,\beta$ $\in\mathbb{R}$,
\begin{equation}
|\alpha| + |\beta| \geq \sqrt{\alpha^2+\beta^2},
\end{equation}
and equality holds only if $\alpha=0$ or $\beta=0$.
\end{lemma}
\begin{proof}
\begin{eqnarray}
\left(|\alpha|+|\beta|\right)^2 &=& \alpha^2+2|\alpha||\beta|+\beta^2,\\
|\alpha|+|\beta| &=& \sqrt{\alpha^2+2|\alpha||\beta|+\beta^2},\\
|\alpha|+|\beta| &=& \sqrt{\alpha^2+\beta^2}+\sqrt{2|\alpha||\beta|},\\
|\alpha|+|\beta| &\geq & \sqrt{\alpha^2+\beta^2}. 
\end{eqnarray}
\end{proof}
Since any complex number $z$ may be expressed by real values $\alpha,\beta$ as $z=\alpha+i\beta$, we may use the lemma to determine whether the complex eigenvalue spectrum is purely imaginary or real. Consider
\begin{equation}
|z|^2 = z\bar z = (\alpha+i\beta)(\alpha-i\beta) = \alpha^2+\beta^2,
\end{equation}
and thus
\begin{equation}
|z| = \sqrt{\alpha^2+\beta^2} \leq |\alpha|+|\beta|.
\end{equation}
Requiring equality yields that
\begin{equation}
\frac{|\alpha|+|\beta|}{\sqrt{\alpha^2+\beta^2}} = 1,
\end{equation}
so that $z$ is either purely imaginary or real if
\begin{equation}
\frac{|\alpha|+|\beta|}{\sqrt{\alpha^2+\beta^2}} - 1 = 0.
\end{equation}
Since a true solution must be purely imaginary or real, we set the $j$-th eigenvalue to be
\begin{equation}
\omega_j = \alpha_j+i\beta_j,
\end{equation}
and evaluate the quality of the eigenvalue solution spectrum with
\begin{equation}
\text{err} = \sum_{j=1}^N \left(\frac{|\alpha_j|+|\beta_j|}{\sqrt{\alpha_j^2+\beta_j^2}} - 1\right).
\end{equation}
Of the three linearization methods described above, the $\mathbf{L3}$ factorization provided the smallest error. This is the method used throughout the analysis.


\chapter{M639 WRITEUP}
\label{M639}

\section{Application: Finite Element Analysis}
In this section, we will develop the finite element formulation of a classical problem: the 1-D Cantilever Beam.

Under the new notation, the element equations of motion can be expressed as
\[
[\textbf{M}]\left\{\ddot{\vec y}_i\right\}+[\textbf{K}]\left\{\vec y_i\right\} = 0,
\]
and noticing that this is a second order ODE without damping or forcing, we assume the ansatz
\[
\left\{\vec y_i\right\} = \left\{\vec \xi_i\right\}\sin(\omega_i t).
\]
The amplitudes $\vec \xi_i$ are independent of time, and $\omega_i$ is the vibration frequency.
Plugging the ansatz into the ODE, we find
\[
(\textbf K-\omega_i^2\textbf M)\vec \xi_i = 0,
\]
which is the \emph{generalized eigenvalue problem}, with eigenvalues $\lambda_i = -\omega_i^2$, and eigenvectors $\vec \xi_i$. The $\vec\xi_i$ are commonly referred to as \emph{eigenshapes}, and represent the steady state displacements of the beam.\newline
