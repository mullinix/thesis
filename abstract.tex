% You insert your abstract in the space below.
Coupled rotating cantilever beams are ubiquitous in technology. From simple fans to turbine engines and electricity producing windmills, they can be found nearly everywhere. This study aims to characterize possible failure points when considering multiple cantilever beam systems coupled around a central hub which is rotating. The energy equations include effects from gyroscopic as well as Coriolis effects and are fully developed from first physical principles and variational calculus. The study sweeps over industrially practical rotation speeds and tests periodically perturbed rotations for harmonic resonances of natural modes. An application, written  in C++, is designed to assemble finite element matrices with consideration of various spatial orientations, node connectivity, external forcing and boundary conditions for generic structures. A second set of programs, written in MatLab, performs multiple analysis routines including static modal analysis as well as forced response analysis via forward time integration. All routines are designed for flexibility and maintainability while keeping efficiency at the core.


