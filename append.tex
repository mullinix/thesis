\appendices
%
% If you only have one appendix, you should change the above to:
%\appendix
%

\chapter{VARIATIONAL APPROXIMATION AND CALCULUS OF VARIATIONS}
\label{app:variations}
In this appendix a treatment on variational approximation is presented as the background and basis for the finite element approximation technique that is employed in solving the cantilever beam problem.

\section{Introduction to Variations}
Let $a,b,\alpha,\beta,\gamma\in\mathbb R$. Suppose there is a function 
\begin{equation}
\eta(x):[a,b]\rightarrow[\alpha,\beta],
\end{equation}
with $\eta\in\mathcal C^1[a,b]$, such that
\begin{equation}
\eta(a) = \eta(b) = 0.
\end{equation}
Suppose there is also a function
\begin{equation}
y(x):[a,b]\rightarrow[\alpha,\beta],
\end{equation}
with $y\in\mathcal C^1[a,b]$, where
\begin{equation}
\begin{array}{rl}
y(a) = \alpha,&y(b) = \beta.
\end{array}
\end{equation}
For the parameter $\gamma$, define:
\begin{equation}
Y(x;\gamma) \equiv y(x)+\gamma\eta(x).
\end{equation}
Clearly, as $\gamma\rightarrow 0$, $Y(x;\gamma)\rightarrow y(x)$. Furthermore, for any $\eta$ and $\gamma$,
\begin{equation}
\begin{array}{rl}
Y(a;\gamma) = y(a),&Y(b;\gamma) = y(b).
\end{array}
\end{equation}
By the chain rule,
\begin{align}
\frac{\text dY}{\text dx} &= \frac{\partial Y}{\partial y}\frac{\text dy}{\text dx}+\gamma\frac{\partial Y}{\partial y}\frac{\text d\eta}{\text dx}\\
						  &= 1\cdot y'+1\cdot \gamma\eta'\\
						  &= y'+\gamma\eta'.
\end{align}
Since $y,\eta$ $\in\mathcal C^1[a,b]$, this result has that $Y$ $\in\mathcal C^1[a,b]$, and we write $Y' = \frac{\text dY}{\text dx}$.\newline
We now define the variation, with operator $\delta$.
\begin{definition} 
Given $Y$ as defined above, we define:
\begin{equation}
\delta Y = Y(x;\gamma)-Y(x;0) = \gamma\eta(x),
\end{equation}
and call this the \emph{\textbf{variation of $Y$}}, and call $\delta$ the \emph{\textbf{variational operator}}.
\end{definition}
From this definition, we see that for $0<\gamma\ll 1$, $\delta Y \approx 0$, and, moreover, 
\begin{equation}
\lim_{\gamma\to 0}{\delta Y}=0.
\end{equation}
In order to understand the linearity of $\delta$, observe that:
\begin{align}
\frac{\text d}{\text dx}\left[\delta Y\right] &= \frac{\text dY(x;\alpha)}{\text dx}-\frac{\text dY(x;0)}{\text dx}\\
                                 &= y'+\gamma\eta' - y'\\
                                 &= \gamma\eta',
\end{align}
and
\begin{align}
\delta\left[\frac{\text dY}{\text dx}\right] &= \delta\left[y'+\gamma\eta'\right]\\
                                 &= y'+\gamma\eta' - y'\\
                                 &= \gamma\eta',
\end{align}
so clearly
\begin{equation}
\frac{\text d}{\text dx}\left[\delta Y\right] = \delta\left[\frac{\text dY}{\text dx}\right].
\end{equation}
Similarly,
\begin{align}
\delta\int{Y}\text dx &= \int{Y(x;\gamma)}\text dx-\int{Y(x;0)}\text dx\\
					  &= \int{Y(x;\gamma)-Y(x;0)}\text dx\\
					  &= \int{\delta Y}\text{ d}x.
\end{align}
A full treatment of this property has been presented in Refs.~\cite{cornelius1970variational,weinstock1974calculus}.
\begin{definition}
Let $y$ be defined as above. Consider the following integral:
\begin{equation}
J = \int_a^b{f(x,y,y')}\text{ d}x,
\end{equation}
for some function $f$ in $\mathcal C^1[a,b]$. Suppose that there is some function $y$ which minimizes $J$, where the function $y$ is unknown. Suppose furthermore that there is some other function $Y$ which is known and is ``near'' $y$. In order to find $y$, we construct $Y$ as above, and write
\begin{equation}
\hat J(\gamma) = \int_a^b{f(x,Y,Y')}\text{ d}x.
\end{equation}
$\hat J$ is called a \emph{\textbf{functional}}, and $\gamma=0\Rightarrow Y=y$ extremizes $\hat J$ in the function space. 
\end{definition}
\begin{definition}
Consider:
\begin{equation}
\hat J' = \frac{\text{ d}\hat J}{\text{ d} \gamma} = \int_a^b{\frac{\partial f}{\partial Y}\frac{\partial Y}{\partial \gamma}+\frac{\partial f}{\partial Y'}\frac{\partial Y'}{\partial \gamma}}\text{ d}x = \int_a^b{\frac{\partial f}{\partial Y}\eta+\frac{\partial f}{\partial Y'}\eta'}\text{ d}x.
\end{equation}
Allowing $\gamma\to 0$, we have
\begin{equation}
\hat J'(0) = \int_a^b{\frac{\partial f}{\partial y}\eta-\frac{\partial f}{\partial y'}\eta'}\text{ d}x,
\end{equation}
and integrating by parts provides:
\begin{equation}
 \hat J'(0) = \left.\frac{\partial f}{\partial y'}\eta\right|_a^b +\int_a^b{\left[\frac{\partial f}{\partial y}-\frac{\text{ d}}{\text{ d}x}\left(\frac{\partial f}{\partial y'}\right)\right]}\eta\text{ d}x = \int_a^b{\left[\frac{\partial f}{\partial y}-\frac{\text{ d}}{\text{ d}x}\left(\frac{\partial f}{\partial y'}\right)\right]}\eta\text{ d}x.
 \end{equation}
Since we are minimizing $\hat J$, we set this integral to zero; that is:
\begin{equation}
\int_a^b{\left[\frac{\partial f}{\partial y}-\frac{\text{ d}}{\text{ d}x}\left(\frac{\partial f}{\partial y'}\right)\right]}\eta\text{ d}x = 0,
\end{equation}
and since this requirement must hold for any $\eta$, it must be true that (see Ref.~\cite{weinstock1974calculus}) 
\begin{equation}
\frac{\partial f}{\partial y}-\frac{\text{ d}}{\text{ d}x}\left(\frac{\partial f}{\partial y'}\right) = 0.
\label{eq:euler.lag.def}
\end{equation}
These are known as the \emph{\textbf{Euler-Lagrange Equations}}.
\end{definition}
The term \emph{Euler-Lagrange Equations} is plural because, in spite of the fact that Eq.~(\ref{eq:euler.lag.def}) appears as a single equation, this extends to $n$-dimensional space for $y,y'\in\mathbb R^n$. Thus, Eq.~(\ref{eq:euler.lag.def}) is actually a system of equations. A full treatment of this is given in Ref.~\cite{cornelius1970variational}.

\section{Variational Minimization}
Alternatively, consider
\begin{equation}
\delta\hat J = \int_a^b{\delta f}\text{ d}x,
\end{equation}
and, following from Ref.~\cite{cornelius1970variational},
\begin{equation}
\delta f = f(x,Y(x;0),Y'(x;0))-f(x,Y(x;\gamma),Y'(x;\gamma)) = \gamma\left[\frac{\partial f}{\partial Y}\eta+\frac{\partial f}{\partial Y'}\eta'\right].
\end{equation}
Using this, we write
\begin{align}
\delta\hat J &= \int_a^b{\delta f}\text{ d}x,\\
             &= \gamma\int_a^b{\left[\frac{\partial f}{\partial Y}\eta+\frac{\partial f}{\partial Y'}\eta'\right]}\text{ d}x\\
             &= \gamma\int_a^b{\left[\frac{\partial f}{\partial y}-\frac{\text{ d}}{\text{ d}x}\left(\frac{\partial f}{\partial y'}\right)\right]}\eta\text{ d}x.
\end{align}
Thus, the variation of our functional can be expressed as a scalar multiple of the rate of change of our functional with respect to the parameter $\gamma$. We conclude that the variation of the functional can be used to minimize the functional. It can be shown  that this formulation holds for $n$-dimensional $x$, $Y$, and $Y'$,  Ref.~\cite{cornelius1970variational}.

\section{The Principle of Least Action}
The Euler-Lagrange equations are often used to describe the energy of a system. The more common notation associated with energy is
\begin{equation}
\mathcal L = L(t,x,\dot x),
\end{equation}
and the integral
\begin{equation}
\mathcal S = \int_{t_0}^{t_1}\mathcal L\text{ d}t = \int_{t_0}^{t_1}L(t,x,\dot x)\text{ d}t
\end{equation}
is called the \emph{action}. The Principle of Least Action states that as system will take the path of least action. In this case, we can determine the path of least action by taking the variation of the action functional equal to zero:
\begin{equation}
\delta S = \delta \int_{t_0}^{t_1}L(t,x,\dot x)\text{ d}t = 0,
\end{equation}
or 
\begin{equation}
\int_{t_0}^{t_1}\delta L(t,x,\dot x)\text{ d}t = 0 \Leftrightarrow \frac{\partial L}{\partial x}-\frac{\text{ d}}{\text{ d}t}\left(\frac{\partial L}{\partial \dot x}\right) = 0.
\end{equation}
The \emph{Lagrangian} of a system is given as 
\begin{equation}
\mathcal L = T-V,
\end{equation}
where $T$ is the kinetic energy of the system and $V$ is the potential energy in the system. For this discussion, it shall be assumed that all virtual work is stored as energy in $V$, and will derive the work in its potential form. This is valid for mass-conserving systems and the work is described as the conservative force minus the change in potential energy, which leads immediately to \cite{petyt1990introduction}:
\begin{equation}
\delta W = -\delta V,
\end{equation}
where in this case $W$ represents virtual work.

\section{The Ritz-Rayleigh Method}
While the aforementioned variational approximation is useful for approximating the ``mass action'' of the system, it is not enough for computing true motion of the system. In order to do so, a discretization technique must be employed. Since we know that the energy equations are useful in variational approximation, we are motivated to utilize the energy equations to elucidate an approximation for the equations of motion. We shall employ Ritz's method to approximate the system so that we may calculate the discrete equations of motion. In essence, Ritz's method provides us with an appropriate solution, and in doing so, yields a discretized system. By construction, the error (or residual) is minimized, and the boundary values are all accounted for. In finite difference techniques, a template is applied to a problem, and stability, well-posedness, numerical and system boundary conditions all need to be computed and verified. Finite element methods, including Ritz's method, account for all of these ingredients, so long as appropriate test functions are chosen \cite{axelsson1984finite}.

The immediate question to be posed is: ``How does one choose appropriate test functions?'' The answer, however, is not nearly straightforward. In practice, it is easy to determine the appropriate function space and calculate the test functions. In terms of proving that the test functions are appropriate, the theory is quite complicated. It is beyond the scope of this paper, and shall not be thoroughly discussed. A full treatment can be found in Brenner's text \cite{brenner2008mathematical}. The foundations, however, shall be stated.

\begin{definition}
Let $f\in\mathcal C^1[a,b]$ be such that
\begin{equation}
\int_a^b\left|f(x)\right|^n\text{ d}x<\infty,
\label{eq:leb.int}
\end{equation}
and define
\begin{equation}
||f||_{n} = \left(\int_a^b\left|f(x)\right|^n\text{ d}x\right)^{1/n}
\end{equation}
as the $n$-th $L$-norm. Then all $f$ meeting the requirement in Eqn.~\ref{eq:leb.int}, coupled with the corresponding $L$-norm, forms the \emph{$n$-th \textbf{Hilbert space, $H^n([a,b])$}}. That is, 
\begin{equation}
H^n([a,b]) = \left\{f\in\mathcal C^1[a,b]:\int_a^b\left|f(x)\right|^n\text{ d}x<\infty\right\},
\end{equation}
with corresponding $L$-norm. 
\end{definition}
Of particular interest, for the finite element application, is $H^2$.

\emph{Note:} This is a \emph{particular} definition of a Hilbert space. There are more generic definitions, but this definition is quite suitable for our application. For more information on Hilbert spaces, and other norms/normed spaces, refer to Ref.~\cite{brenner2008mathematical}.

We now consider an extension of above ideas. 
\begin{definition}
Let
\begin{equation}
Y_n(x;\vec\gamma) = y(x)+\sum_{k=1}^n\gamma_k\eta_k(x),
\end{equation}
where $\eta_i$, $i=1,\dots,n$ are as above --- \emph{i.e.} $\eta_i(a) = \eta_i(b) = 0$ for each $i$. Let 
\begin{equation}
J = \int_a^b{f(x,Y_n,Y_n')}\text{ d}x.
\end{equation}
Then 
\begin{equation}
\delta J = 0 \Leftrightarrow \frac{\partial f}{\partial Y_n}-\frac{\text{ d}}{\text{ d}x}\left(\frac{\partial f}{\partial Y_n'}\right) = 0.
\end{equation}
If this is satisfied, then $J$ is minimized by $Y^*$, say, and there must be some $Y_n^*$ comprised of the resulting solution parameters $\vec\gamma^*$ that satisfies
\begin{equation}
||Y^*-Y_n^*||_2\leq||Y^*-V||_2\text{, for any $V$ $\in H^2([a,b]).$}
\end{equation}
This is known as \emph{\textbf{Ritz-Rayleigh minimization}}.
\end{definition}
This may not look very useful, but indeed we have the result that a \emph{finite} dimensional function can be used to approximate an \emph{infinite} dimensional problem. And, we have information about how to pick such finite dimension functions --- they need to be in $H^2$ and they must satisfy the boundary criteria. Thus, we may use well known and characterized approximation functions to discretize our system, and at the same time, we will have found equations of motion of our system from the energy functions.

Perhaps more interestingly, what this statement says is that if we can find a function $Y^*_n$ which satisfies the criteria --- it meets the boundary conditions and is in $H^2$, then providing that the function is consistent with the problem (\emph{i.e.} is periodic for periodic systems, etc.) --- then we can be assured that the test function is as close to the minimizing function as any other function in the function space.

\emph{Note:} Not all problems can be solved using Ritz's method. The problem must have a variational formulation; in other words, it must satisfy the Euler-Lagrange equations. If not, such as the case of non-conservative work, the variation cannot be used to minimize the functional, and the problem must be solved using Galerkin's method of the weighted residual. See Ref.~\cite{duchateau1992advanced} for a description of this method.

One final observation can be made. The Ritz method does not need to be applied when discretizing a system that is in variational form. Any of the Galerkin weighted residual methods will also work; however, when applying those methods to a problem in variational form, the result will always reduce to the same form as the Ritz approximation as discussed in Ref.~\cite{reddy1984energy}. This is important to note because many authors publish that their approximation is done via weighted methods, but in the end, given that their system is in variational form, the final form will be the same as the resultant form of the Ritz approximation method. Furthermore, the notation behind weighted methods is sometimes simpler, and for that reason, one might be compelled to setup the problem in a weighted form; this result allows for such a substitution.

\chapter{STRAIN EQUATIONS}
\label{app:strain}
This appendix derives the strain equations from first principles of structural mechanics for use in building the stress and strain terms in the cantilever beam problem.

\section{Normal Strain}
The strain energy of the beam is proportional to the deformation and deforming force. Mathematically, this is expressed as
\begin{equation}
U = \int_0^\delta F\text{ d}\delta,
\label{eq:strain.base}
\end{equation}
where $U$ is strain energy, $F$ is the applied force, and $\delta$ is the displacement.

The \textbf{average} \emph{Normal strain} is given by 
\begin{equation}
\varepsilon_{\text{avg}} = \frac{\Delta L}{L},
\label{eq:avg.normal.strain}
\end{equation}
where $L$ is the length and $\Delta$ is the usual difference operator, and the average normal strain is a dimensionless ratio. This strain is normal to the applied force.

\begin{figure}[ht!]
\caption{Normal strain in a 1D rod.}
\centering
\includegraphics[width=0.65\textwidth]{images/normal_strain.pdf}
\label{im:norm.strain}
\end{figure}

The \textbf{point} normal strain in 1D is constructed as follows. Suppose two points are examined along a rod as in Figure \ref{im:norm.strain}, say $P$ and $Q$. The distance between the points is denoted 
\begin{equation}
\Delta x = Q-P,
\end{equation} 
and the displacements of the two points $P$ and $Q$ are denoted as $u_P$ and $u_Q$ so that the displacement is given as 
\begin{equation}
\Delta u = u_Q-u_P.
\end{equation}
By setting 
\begin{equation}
u = u_P
\end{equation}
and adding zero
\begin{equation}
u_Q = u_Q + u_P - u_P = u_P+(u_Q-u_P)
\end{equation}
we find
\begin{equation}
u_Q = u + \Delta u,
\end{equation}
and we can write the difference quotient
\begin{equation}
\frac{(u+\Delta u)-u}{\Delta x}.
\end{equation}
The point normal strain is given as the limit of this quotient
\begin{equation}
\varepsilon_p = \lim_{\Delta x\to 0} \frac{(u+\Delta u)-u}{\Delta x} = \lim_{\Delta x\to 0} \frac{\Delta u}{\Delta x} = \frac{\text{ d}u}{\text{ d}x}
\end{equation}

In the case of higher dimensions, the displacement functions are functions of higher dimension, and the point normal strain in each direction would be a partial derivative, \emph{e.g.}
\begin{equation}
\varepsilon_{xx} = \frac{\partial u}{\partial x}
\label{eq:strain.oned}
\end{equation}
for strain in the $x$-direction. The subscript, in this case, is given in \emph{index} notation for the displacement vector $u$ and the spatial displacement in $x$. 

The mixed case is more general, and is a combination of multiple terms, given as
\begin{equation}
\label{eq:strain.linear}
\varepsilon_{ij} = \frac{1}{2}\left\lbrace \frac{\partial u_i}{\partial x_j}+\frac{\partial u_j}{\partial x_i}\right\rbrace +\frac{1}{2}\left\lbrace \sum_{k=1}^3{\frac{\partial u_k}{\partial x_i}\frac{\partial u_k}{\partial x_j}}\right\rbrace
\end{equation}
in the orthogonal $ijk$ coordinate frame.

It is important to note that this relationship only holds for small strains. The linearity of the displacement-strain relationship is a local effect. Large strains often result in nonlinear deformations. This requires consideration of the higher order terms, which yields \emph{e.g.}
\begin{equation}
\label{eq:strain.hot}
\varepsilon_{xx} = \frac{\partial u}{\partial x}+\frac{1}{2}\left\lbrace \left(\frac{\partial u}{\partial x}\right)^2+\left(\frac{\partial v}{\partial x}\right)^2 +\left(\frac{\partial w}{\partial x}\right)^2\right\rbrace
\end{equation}
for a 3D system.

\section{Shear Strain}
The \emph{Shear strain} is an angular change. It is often thought of as the change in angle between two vectors. The study of shear strain involves the consideration of an infinitesimal volume that is cubic in shape. By considering only one of the faces of the cube, it is possible to discuss the angular relationship between the bottom and left sides of the cube. Before deformation, they are $\pi/2$ radians apart; once deformed, they will be $\pi/2-\gamma$ radians apart, where $\gamma = \gamma_1+\gamma_2$ is the final angular deformation. Here $\gamma_1$ and $\gamma_2$ represent, respectively, the angular displacements of the bottom and left edges with respect to their initial positions. 
\begin{figure}[ht!]
\caption[Shear strain over a square.]{Shear strain over a square. \protect\bibentry{boulderstrain4}}
\centering
\includegraphics[width=0.65\textwidth]{images/ucboulderstrain4.png}
\label{im:strain.boulder}
\end{figure}

The \emph{Average shear strain} is given as
\begin{equation}
\gamma_{\text{avg}} = \frac{\pi}{2}-\gamma,
\end{equation}
where $\gamma$ is the total angular displacement of the edges.

In general, it is also possible for the cube to be translated from its original position, and thus a general form will need to be implemented to find the displacements $\gamma_1$ and $\gamma_2$. From Figure \ref{im:strain.boulder}, we consider the right triangle formed with hypotenuse $A'B'$. The angle $\gamma_1$ is known to be the arctangent of the vertical over the horizontal legs. Using the corresponding notation, we have
\begin{equation}
\tan{(\gamma_1)} = \left(\frac{v_B-v_A}{u_B-u_A}\right)
\label{eq:strain.atan}
\end{equation}
and we write the numerator as a difference, while noting that the denominator is a change in the distance of segment $AB$. For small strains, this change will be trivial, so we approximate
\begin{equation}
u_B-u_A\approx\Delta x.
\label{eq:strain.approx}
\end{equation}
Plugging Eq.~\ref{eq:strain.approx} into Eq.~\ref{eq:strain.atan}, and writing the numerator as a difference, we have
\begin{equation}
\tan{(\gamma_1)} = \left(\frac{\Delta v_{BA}}{\Delta x}\right).
\end{equation}
Furthermore, by assuming small strains we have
\begin{equation}
\label{eq:shear.assump}
\gamma_1\ll1
\end{equation}
so that 
\begin{equation}
\tan{(\gamma_1)}\approx\gamma_1.
\end{equation}

Following a similar process for $\gamma_2$, we construct
\begin{equation}
\gamma_1 = \frac{\Delta v}{\Delta x},\qquad\gamma_2 = \frac{\Delta u}{\Delta y},
\end{equation}
and write
\begin{equation}
\gamma = \gamma_1+\gamma_2 = \frac{\Delta v}{\Delta x}+\frac{\Delta u}{\Delta y}.
\end{equation}
We now write the definition of the \emph{Point shear strain} as
\begin{equation}
\label{eq:shear.linear}
\gamma_{xy} = \lim_{\Delta x,\Delta y\to 0}\gamma = \lim_{\Delta x,\Delta y\to 0}\frac{\Delta v}{\Delta x}+\frac{\Delta u}{\Delta y} = \frac{\partial v}{\partial x} + \frac{\partial u}{\partial y}.
\end{equation}
Linearity of $\partial$ implies that $\gamma_{xy}=\gamma_{yx}$.

This can be easily extended to 3D so that all faces of the cube can be analyzed for shear strain.

\section{Strain Energy Density}
Average strain energy density is given as
\begin{equation}
\bar U_d = \int \sigma \text{ d}\varepsilon,
\label{eq:strain.density}
\end{equation}
where $\sigma$ is \textbf{stress}, and $\varepsilon$ is \textbf{strain}. \emph{Normal stress} is defined as
\begin{equation}
\sigma = \frac{F}{A},
\label{eq:stress.force}
\end{equation}
where $F$ is the applied force, and $A$ is the surface area upon which the force is applied. 
\emph{Shear stress} is defined as
\begin{equation}
\uptau = \frac{T}{A},
\end{equation}
where $T$ is the applied torque, and $A$ is the surface area upon which the torque is applied. $F$ is a force normal to the cross section, while $T$ is tangential to the cross section.

\emph{Hooke's Law} relates stress to strain by the linear relationship
\begin{equation}
\sigma = E\varepsilon,
\label{eq:strain.hooke}
\end{equation}
where $E$ represents Young's modulus, or the elastic modulus, and is a material property. Hooke's law holds for a small range on most materials, but some materials are nonlinear and do not have a linear relationship. For our purposes, the materials are linear for small strains; this is true for many common metallic alloys such as aluminum and steel \cite{boulderstrain5}.

Plugging Eq.~\ref{eq:strain.hooke} into Eq.~\ref{eq:strain.density} yields
\begin{equation}
\bar U_d = \int E\varepsilon\text{ d}\varepsilon = \frac{1}{2} E\varepsilon^2 = \frac{1}{2}\sigma\varepsilon = \frac{1}{2}\left(\frac{\sigma^2}{E}\right).
\label{eq:strain.energy.final}
\end{equation}

Total strain energy of a rod or beam is given by
\begin{equation}
U = \int_V\bar U_d \text{ d}V = \int_0^L\int_A\frac{1}{2}\left(\frac{\sigma^2}{E}\right)\text{ d}A\text{ d}x.
\end{equation}

Similar to Hooke's law, there is a linear relationship between shear stress and shear strain given by
\begin{equation}
\uptau = G\gamma,
\end{equation}
where $\gamma$ is shear strain, $G$ is the material property called shear modulus, and $\uptau$ is shear stress. The same procedure as above can be applied to calculate the strain energy from shear.

\emph{Poisson's Ratio} is the absolute value of the lateral strain over the axial strain, is a dimensionless value, and is denoted by $\nu$. $E$, $G$, and $\nu$ are related by
\begin{equation}
E = 2G(1+\nu).
\end{equation}

\chapter{LINEARITY ASSUMPTIONS}
\label{app:assump}
This appendix tabulates the various linearizing assumptions for reference. The results are stored in Table~\ref{tbl:linearization}. Subscripts indicate differentiation with respect to the subscript label (except the case of $h_v$ and $h_w$). The phrase ``higher order terms'' is abbreviated as ``h.o.t.''.
\begin{table}
\caption[Linearizing Assumptions from Development of the Equations of Motion]{Linearizing Assumptions from Development of the Equations of Motion}
\label{tbl:linearization}
\begin{tabular}{|c|c|c|}
\hline 
Reference & Simplification & Explanation \\ 
\hline 
Strain: Eqs.~\ref{eq:strain.linear}~and~\ref{eq:strain.hot} & $u_x+0.5u_x^2\rightarrow u_x$ & 
\begin{tabular}{c}
Small deformations corres-\\ 
pond to small strains
\end{tabular} \\ 
\hline 
Shear: Eqs.~\ref{eq:shear.assump}~and~\ref{eq:shear.linear} & \begin{tabular}{c} $\gamma\ll 1\Rightarrow\tan(\gamma)\approx\gamma$\\
$\rightarrow \gamma_{xy}=v_x+u_y$
\end{tabular}
& \begin{tabular}{c}
Small deformations corres-\\
pond to small shear strains
\end{tabular} \\
\hline
Stretch: Eqs.~\ref{eq:arclength.u.assump}~and~\ref{eq:arclength.u.approx} & \begin{tabular}{c}
$1+2u_\phi\gg u_\phi^2$ \\
$\rightarrow(1+u_\phi)^2\approx 1+2u_\phi$
\end{tabular}
& \begin{tabular}{c}
Small deformations corres-\\
pond to small stretch
\end{tabular}
\\
\hline
Stretch: Eq.~\ref{eq:f_func.approx} 
& \begin{tabular}{c}
$y^2\ll y$ \\
$\rightarrow f(y)\approx 1+0.5y$
\end{tabular} 
& \begin{tabular}{c}
Small deformations corres-\\
pond to small stretch
\end{tabular}
\\
\hline
Curvature: Eq.~\ref{eq:radius.curvature} 
& \begin{tabular}{c}
$y'\ll 1$\\
$\rightarrow \rho=1/y_{xx}$
\end{tabular}
& \begin{tabular}{c}
Small deformations corres-\\
pond to small bending
\end{tabular}
\\
\hline
\begin{tabular}{c}
Linearize the EOM\\
Eqs.~\ref{eq:s.full.nonlinear}~and~\ref{eq:EL_s_final}
\end{tabular}
& $-\rho A(\ddot h_v+\ddot h_w)\rightarrow 0$
& \begin{tabular}{c}
Removing h.o.t.\\
$\ddot h_v$ and $\ddot h_w$ are nonlin.
\end{tabular}
\\
\hline
\begin{tabular}{c}
Linearize the EOM\\
Eqs.~\ref{eq:v.full.nonlinear}~and~\ref{eq:EL_v_final}
\end{tabular}
& \begin{tabular}{c}
$\rho A\left\{\left[\int_L^x\Omega^2(s-h_v-h_w)\right.\right.$\\
$\left.\left.\vphantom{\frac{0}{0}}+\dot v\Omega\text{ d}\eta\right]v_x\right\}_x\rightarrow 0,$\\
$-2\Omega(\dot h_v+\dot h_w)\rightarrow 0$
\end{tabular}
& \begin{tabular}{c}
Removing h.o.t.\\
Products on $v_x$ are nonlin.\\
$\dot h_v$ and $\dot h_w$ are nonlin.
\end{tabular}
\\
\hline
\begin{tabular}{c}
Linearize the EOM\\
Eqs.~\ref{eq:w.full.nonlinear}~and~\ref{eq:EL_w_final}
\end{tabular}
& \begin{tabular}{c}
$\rho A\left\{\left[\int_L^x\Omega^2(s-h_v-h_w)\right.\right.$\\
$\left.\left.\vphantom{\frac{0}{0}}+\dot v\Omega\text{ d}\eta\right]w_x\right\}_x\rightarrow 0$
\end{tabular}
& \begin{tabular}{c}
Removing h.o.t.\\
Products on $w_x$ are nonlin.
\end{tabular}
\\
\hline
\end{tabular} 
\end{table}
